\documentclass{article}

\usepackage[colorlinks,bookmarksopen,bookmarksnumbered,citecolor=red,urlcolor=red]{hyperref}
\usepackage{moreverb}
\usepackage{amsmath}
\usepackage{amssymb}
\usepackage[margin=1.0in]{geometry}

\begin{document}

\title{A summary of the eqations of motion on the cubed-sphere under the shallow-atmosphere and spherical geopotential approximations}

\author{Martin Charron, St\'ephane Gaudreault \\
\\ \vspace{6pt} Recherche en pr\'evision num\'erique atmosph\'erique, 
\\ Environnement Canada, Dorval, Qc, Canada}

\maketitle

\begin{abstract}
The governing equations in the cubed-sphere horizontal coordinates with a generalized vertical coordinate are written using the four-dimensional tensor formalism. The shallow-atmosphere and non-hydrostatic equations under the spherical geopotential approximation are considered.
\end{abstract}

\section{Equations of motion}
\subsection{Euler equations}
The inviscid horizontal momentum equations are ($i=1,2$):
\begin{align}
\left( \sqrt{g}\rho u^\nu u^i\right)_{,\nu} = - \sqrt{g}h^{ij}p_{,j} -2\sqrt{g} \rho \Gamma^i_{j0} u^j - \sqrt{g} \rho \Gamma^i_{jk}u^ju^k,
\end{align}
while the inviscid vertical momentum equation with one vertical velocity is
\begin{align}
\left( \sqrt{g}\rho u^\nu u^3\right)_{,\nu} = - \sqrt{g} \rho \left(\frac{\partial z}{\partial \eta}\right)^{-1} g_r - \sqrt{g}h^{3j}p_{,j} -2\sqrt{g} \rho \Gamma^3_{j0} u^j - \sqrt{g} \rho \Gamma^3_{jk}u^ju^k.
\end{align}
The independent variable $x^3=\eta$ is a generalized vertical coordinate and $z$ is height above the spherical Earth of radius $a$. Therefore, $u^3=\dot\eta$. The constant $g_r$ is approximately $9.81$ m s$^{-2}$.

If two vertical velocities are chosen instead of one, then $u^3$ is diagnosed from the other vertical velocity ($w=\dot z$):
\begin{align}
u^3=\left(\frac{\partial z}{\partial \eta}\right)^{-1}\left[ w - \frac{\partial z}{\partial x^1} u^1 - \frac{\partial z}{\partial x^2} u^2 \right]
\end{align}
and the vertical momentum equation is written:
\begin{align}
\left(\sqrt{g}\rho u^\nu w\right)_{,\nu}=-\sqrt{g}\rho g_r-\sqrt{g} \left(\frac{\partial z}{\partial \eta}\right)^{-1} \frac{\partial p}{\partial\eta}.
\end{align}

The continuity equation takes the form
\begin{align}
\left( \sqrt{g}\rho u^\nu \right)_{,\nu} = 0.
\end{align}

\subsection{Shallow water equations}
For the shallow water equations, we use the Einstein summation rules for repeated lower and upper Greek indices ranging from 0 to 2 and for repeated lower and upper Latin indices from 1 to 2.

The momentum equations are ($i=1,2$):
\begin{align}
   \left( \sqrt{g} H u^\nu u^i\right)_{,\nu} = - \left( \sqrt{g} \, \frac{1}{2} \, g_r \, h^{ij} (H)^2 \right)_{,j} -2\sqrt{g} H \Gamma^i_{j0} u^j - \sqrt{g} H \Gamma^i_{jk}u^ju^k - \sqrt{g} \, g_r H \left( h^{ij} \dfrac{\partial z}{\partial x^j} \right)
\end{align}
where $H$ denotes the height of the fluid and $z$ is the bottom topography.

The continuity equation takes the form
\begin{align}
\left( \sqrt{g} H u^\nu \right)_{,\nu} = 0.
\end{align}


\section{Geometrical terms}
Define $X=\tan x^1$, $Y=\tan x^2$, and $\delta^2=1+X^2+Y^2$ on all six panels. The non-zero components of the tensor $h^{\mu \nu}$ are
\begin{align}
h^{11}&=\frac{\delta^2}{a^2(1+X^2)}, \\
h^{12}&=\frac{XY\delta^2}{a^2(1+X^2)(1+Y^2)}=h^{21}, \\
h^{13}&=-\left( \frac{\partial z}{\partial \eta} \right)^{-1} \left( \frac{\partial z}{\partial x^1} h^{11} + \frac{\partial z}{\partial x^2} h^{12} \right)=h^{31}, \\
h^{22}&=\frac{\delta^2}{a^2(1+Y^2)}, \\
h^{23}&=-\left( \frac{\partial z}{\partial \eta} \right)^{-1} \left( \frac{\partial z}{\partial x^1} h^{12} + \frac{\partial z}{\partial x^2} h^{22} \right)=h^{32}, \\
h^{33}&=\left( \frac{\partial z}{\partial \eta} \right)^{-2} \left[ 1 + \left( \frac{\partial z}{\partial x^1} \right)^2 h^{11} + \left( \frac{\partial z}{\partial x^2} \right)^2 h^{22} +2\frac{\partial z}{\partial x^1} \frac{\partial z}{\partial x^2} h^{12}\right].
\end{align}
One may then find
\begin{align}
\sqrt{g}=\frac{a^2 (1+X^2)(1+Y^2)}{\delta^3}\left|\frac{\partial z}{\partial\eta}\right|. \label{rgt}
\end{align}
The tensor $h^{\mu\nu}$ and the quantity $\sqrt{g}$ have the same form on all six panels.

On panels $p=1,2,3,4$,
\begin{align}
X&=\tan x^1 = \tan \left( \lambda -(p-1)\frac{\pi}{2} \right), \\
Y&=\tan x^2 = \frac{\tan \phi}{\cos \left( \lambda -(p-1)\frac{\pi}{2}\right)}.
\end{align}

The panel covering the Arctic region is called panel $5$. On panel $5$,
\begin{align}
X=\tan x^1&= \frac{\sin \lambda}{\tan \phi}, \\
Y=\tan x^2&=- \frac{\cos \lambda}{\tan \phi}.
\end{align}

The panel covering Antarctica is called panel $6$. On panel $6$,
\begin{align}
X=\tan x^1&=- \frac{\sin \lambda}{\tan \phi}, \\
Y=\tan x^2&=- \frac{\cos \lambda}{\tan \phi}.
\end{align}

For all 6 panels, the Christoffel symbols of the second kind with components $i$ and $j$ are
\begin{align}
\Gamma^1_{11}&=\frac{2 X Y^2 }{ \delta^2 }, \\
\Gamma^1_{12}&=-\frac{ Y(1 + Y^2) }{ \delta^2 }=\Gamma^1_{21}, \\
\Gamma^1_{13}&=0=\Gamma^1_{31}, \\
\Gamma^1_{22}&=0, \\
\Gamma^1_{23}&=0=\Gamma^1_{32}, \\
\Gamma^1_{33}&=0,
\end{align}
\begin{align}
\Gamma^2_{11}&=0, \\
\Gamma^2_{12}&=-\frac{X(1+X^2)}{\delta^2}=\Gamma^2_{21}, \\
\Gamma^2_{13}&=0=\Gamma^2_{31}, \\
\Gamma^2_{22}&=\frac{2X^2Y}{\delta^2}, \\
\Gamma^2_{23}&=0=\Gamma^2_{32}, \\
\Gamma^2_{33}&=0,
\end{align}
\begin{align}
\Gamma^3_{11}&=\left(\frac{\partial z}{\partial \eta}\right)^{-1}\left[ \frac{\partial^2 z}{\partial x^1 \partial x^1} - \frac{2 X Y^2 }{ \delta^2 } \frac{\partial z}{\partial x^1}\right], \\
\Gamma^3_{12}&=\left(\frac{\partial z}{\partial \eta}\right)^{-1}\left[ \frac{\partial^2 z}{\partial x^1 \partial x^2} + \frac{ Y(1 + Y^2) }{ \delta^2 } \frac{\partial z}{\partial x^1} + \frac{X(1+X^2)}{\delta^2} \frac{\partial z}{\partial x^2}   \right]=\Gamma^3_{21}, \\
\Gamma^3_{13}&=\left(\frac{\partial z}{\partial \eta}\right)^{-1} \frac{\partial^2 z}{\partial x^1 \partial \eta}=\Gamma^3_{31}, \\
\Gamma^3_{22}&=\left(\frac{\partial z}{\partial \eta}\right)^{-1}\left[ \frac{\partial^2 z}{\partial x^2 \partial x^2} - \frac{2 X^2 Y }{ \delta^2 } \frac{\partial z}{\partial x^2}\right], \\
\Gamma^3_{23}&=\left(\frac{\partial z}{\partial \eta}\right)^{-1} \frac{\partial^2 z}{\partial x^2 \partial \eta}=\Gamma^3_{32}, \\
\Gamma^3_{33}&=\left(\frac{\partial z}{\partial \eta}\right)^{-1} \frac{\partial^2 z}{\partial \eta^2}.
\end{align}

For panels $1$ to $4$, one calculates
\begin{align}
\Gamma^1_{01}&=\frac{\Omega XY^2}{\delta^2}=\Gamma^1_{10}, \\
\Gamma^1_{02}&=-\frac{\Omega Y(1+Y^2)}{\delta^2}=\Gamma^1_{20}, \\
\Gamma^1_{03}&=0=\Gamma^1_{30}, \\
\Gamma^2_{01}&=\frac{\Omega Y(1+X^2)}{\delta^2}=\Gamma^2_{10}, \\
\Gamma^2_{02}&=-\frac{\Omega XY^2}{\delta^2}=\Gamma^2_{20}, \\
\Gamma^2_{03}&=0=\Gamma^2_{30}, \\
\Gamma^3_{01}&=-\left(\frac{\partial z}{\partial \eta}\right)^{-1}\left[ \frac{\Omega XY^2}{\delta^2} \frac{\partial z}{\partial x^1}+ \frac{\Omega Y(1+X^2)}{\delta^2} \frac{\partial z}{\partial x^2}  \right]=\Gamma^3_{10}, \\
\Gamma^3_{02}&=\left(\frac{\partial z}{\partial \eta}\right)^{-1}\left[ \frac{\Omega Y(1+Y^2)}{\delta^2} \frac{\partial z}{\partial x^1}+\frac{\Omega XY^2}{\delta^2}\frac{\partial z}{\partial x^2}  \right]=\Gamma^3_{20}, \\
\Gamma^3_{03}&=0=\Gamma^3_{30}.
\end{align}
For panel $5$, one calculates
\begin{align}
\Gamma^1_{01}&=\frac{\Omega XY}{\delta^2}=\Gamma^1_{10}, \\
\Gamma^1_{02}&=-\frac{\Omega (1+Y^2)}{\delta^2}=\Gamma^1_{20}, \\
\Gamma^1_{03}&=0=\Gamma^1_{30}, \\
\Gamma^2_{01}&=\frac{\Omega (1+X^2)}{\delta^2}=\Gamma^2_{10}, \\
\Gamma^2_{02}&=-\frac{\Omega XY}{\delta^2}=\Gamma^2_{20}, \\
\Gamma^2_{03}&=0=\Gamma^2_{30}, \\
\Gamma^3_{01}&=-\left(\frac{\partial z}{\partial \eta}\right)^{-1}\left[ \frac{\Omega XY}{\delta^2} \frac{\partial z}{\partial x^1}+ \frac{\Omega (1+X^2)}{\delta^2} \frac{\partial z}{\partial x^2}  \right]=\Gamma^3_{10}, \\
\Gamma^3_{02}&=\left(\frac{\partial z}{\partial \eta}\right)^{-1}\left[ \frac{\Omega (1+Y^2)}{\delta^2} \frac{\partial z}{\partial x^1}+\frac{\Omega XY}{\delta^2}\frac{\partial z}{\partial x^2}  \right]=\Gamma^3_{20}, \\
\Gamma^3_{03}&=0=\Gamma^3_{30}.
\end{align}
For panel $6$, one calculates
\begin{align}
\Gamma^1_{01}&=-\frac{\Omega XY}{\delta^2}=\Gamma^1_{10}, \\
\Gamma^1_{02}&=\frac{\Omega (1+Y^2)}{\delta^2}=\Gamma^1_{20}, \\
\Gamma^1_{03}&=0=\Gamma^1_{30}, \\
\Gamma^2_{01}&=-\frac{\Omega (1+X^2)}{\delta^2}=\Gamma^2_{10}, \\
\Gamma^2_{02}&=\frac{\Omega XY}{\delta^2}=\Gamma^2_{20}, \\
\Gamma^2_{03}&=0=\Gamma^2_{30}, \\
\Gamma^3_{01}&=\left(\frac{\partial z}{\partial \eta}\right)^{-1}\left[ \frac{\Omega XY}{\delta^2} \frac{\partial z}{\partial x^1}+ \frac{\Omega (1+X^2)}{\delta^2} \frac{\partial z}{\partial x^2}  \right]=\Gamma^3_{10}, \\
\Gamma^3_{02}&=-\left(\frac{\partial z}{\partial \eta}\right)^{-1}\left[ \frac{\Omega (1+Y^2)}{\delta^2} \frac{\partial z}{\partial x^1}+\frac{\Omega XY}{\delta^2}\frac{\partial z}{\partial x^2}  \right]=\Gamma^3_{20}, \\
\Gamma^3_{03}&=0=\Gamma^3_{30}.
\end{align}

\end{document}
