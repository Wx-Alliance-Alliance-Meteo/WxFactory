\documentclass{article}

\usepackage{algorithm}
\usepackage{algpseudocodex}
\usepackage{amsmath}
\usepackage{amsfonts}
\usepackage{graphicx}
\usepackage{hyperref}
\usepackage{mathalpha}
\usepackage{tcolorbox}
\usepackage{xcolor}

\newcommand{\todo}[1]{\textcolor{blue}{#1}}



\usepackage{bm}
\newcommand{\varsymbol}{q}
\newcommand{\vars}{\bm{\varsymbol}}
\newcommand{\jacobian}{\bm{J}}
\newcommand{\rhs}{\bm{F}}
\newcommand{\rhssub}[1]{\rhs(\vars_{#1})}
\newcommand{\I}{\bm{I}}
\newcommand{\dt}{\Delta t}
\newcommand{\pseudodt}{\dt^*}
\newcommand{\pseudocfl}{CFL^*}
\newcommand{\systemmat}{\bm{A}}
\newcommand{\systemmatlong}[1]{\left( \I - \dfrac{\dt}{2}\jacobian_{#1} \right)}
\newcommand{\systemmatconv}{\bm{A}_c}
\newcommand{\systemmatvisc}{\bm{A}_v}
\newcommand{\unknown}{\bm{x}}
\newcommand{\fgmres}{\texttt{FGMRES}}
\newcommand{\krylovstart}{\bm{v}}
\newcommand{\gmresprecondletter}{P}
\newcommand{\gmresprecond}{\mathcal{\gmresprecondletter}}
\newcommand{\gmrespremat}{\bm{\gmresprecondletter}}
\newcommand{\smoother}{\mathcal{S}}
\newcommand{\smootherrhs}{\bm{G}}
\newcommand{\smootherrhsconv}{\bm{G}_c}
\newcommand{\smootherrhsvisc}{\bm{G}_v}
\newcommand{\smoothervarsymbol}{u}
\newcommand{\smvec}{\bm{\smoothervarsymbol}}
\newcommand{\smprecondletter}{B}
\newcommand{\smprecond}{\mathcal{\smprecondletter}}
\newcommand{\smprecondmat}{\bm{\smprecondletter}}
\newcommand{\awmat}{\bm{W}}
\newcommand{\mgsol}{\bm{z}}
\newcommand{\mgin}{\bm{y}}
\newcommand{\mgres}{\bm{r}}
\newcommand{\restrict}{\mathcal{R}}
\newcommand{\prolong}{\restrict^{-1}}
\newcommand{\uppermat}{\bm{U}}
\newcommand{\lowermat}{\bm{L}}
\newcommand{\diag}{\bm{D}}
\newcommand{\resrhs}{\bm{b}}
\newcommand{\fluxjac}{\bm{K}}
\newcommand{\elemjac}{\mathcal{J}}
\newcommand{\elemevecr}{\mathcal{E}}
\newcommand{\elemevecl}{{{\elemevecr}^{-1}}}
\newcommand{\elemeval}{\Lambda}
\newcommand{\eigenval}{\lambda}
\newcommand{\soundspeed}{c}
\newcommand{\soundfraction}{d}
\newcommand{\smprecondsolvar}{s}
\newcommand{\smprecondsol}{\bm{\smprecondsolvar}}
\newcommand{\hcontra}[2]{h^{{#1}{#2}}}
\newcommand{\heatcapacityratio}{\gamma}
\newcommand{\pressure}{P}
\newcommand{\density}{\rho}
\newcommand{\gasconstantdry}{R_d}

\usepackage[shortcuts]{glossaries}
\usepackage{booktabs}
\makeglossaries

\glsaddkey{math}{}{\acem}{\Acem}{\acm}{\Acm}{\ACm}

\newacronym[math=\vars]{vars}{Vars}{The set of variables in the equations}
\newacronym[math=\jacobian]{jac}{Jacobian}{Derivative of the RHS w.r.t. variables $\vars$}
\newacronym[math=\rhs]{rhs}{RHS}{Right-hand side of the equations}
\newacronym[math=\I]{I}{Identity}{The identity matrix}
\newacronym[math=\dt]{dt}{dt}{The size of a (real) time step}
\newacronym[math=\pseudodt]{pseudodt}{pseudo dt}{The size of a pseudo time step}
\newacronym[math=\systemmat]{sys_mat}{Sys mat}{System to solve to move forward in time}
\newacronym[math=\unknown]{unknown}{Unknown}{The vector we are trying to solve for with $\systemmat$}
\newacronym[math=\krylovstart]{krylovstart}{Krylov 1}{Starting vector of a Krylov subspace method}
\newacronym[math=\gmresprecond]{precond}{precond}{Preconditioner in \fgmres, applies $\gmrespremat^{-1}$ to its imput}
\newacronym[math=\gmrespremat]{precondmat}{precond mat}{Preconditioner matrix for \fgmres{} ($\gmrespremat\approx\systemmat$)}
\newacronym[math=\smoother]{smoother}{Smoother}{Smoother used as part of the MG algorithm}
\newacronym[math=\mgsol]{mgout}{MG out}{Solution of a multigrid iteration}
\newacronym[math=\mgin]{mgin}{MG in}{Input vector of a MG iteration}
\newacronym[math=\mgres]{mgres}{MG res}{Residual vector (used in MG)}
\newacronym[math=\restrict]{restrict}{restrict}{Restriction operator in MG}
\newacronym[math=\prolong]{prolong}{prolong}{Prolongation operator in MG (inverse of restrict)}
\newacronym[math=\smvec]{smoothervec}{sm vec}{Variable that the smoother solves for}
\newacronym[math=\smprecond]{smprecond}{sm precond}{Preconditioner used by the smoother $\smoother$}
\newacronym[math=\smprecondmat]{smprecmat}{sm pre mat}{Matrix used by the smoother preconditioner $\smprecond$}
\newacronym[math=\awmat]{awmat}{AW mat}{Matrix that represents the implicit RK system to solve? (Additive-W)}
\newacronym[math=\uppermat]{upper}{upper}{Upper triangular part of matrix $\awmat$}
\newacronym[math=\lowermat]{lower}{lower}{Lower triangular part of matrix $\awmat$}
\newacronym[math=\diag]{diag}{diag}{Diagonal part of matrix $\awmat$}
\newacronym[math=\resrhs]{resrhs}{res RHS}{RHS of some system for which we want the residual}
\newacronym[math=\smootherrhs]{smrhs}{sm RHS}{RHS of the smoother pseudo time stepping problem}
\newacronym[math=\fluxjac]{fluxjac}{flux jac}{$5\times 5$ matrix (in 3D) of the flux jacobian in an element}
\newacronym[math=\elemjac]{elemjac}{Eq. jacobian}{Flux jacobian analytic expression for an element}
\newacronym[math=\elemevecr]{elemevecr}{Eq. evec}{Right eigenvectors analytic expression for an element}
\newacronym[math=\elemevecl]{elemevecl}{Eq. evec}{Left eigenvectors analytic expression for an element}
\newacronym[math=\elemjac]{eqjac}{Eq. jacobian}{Flux jacobian analytic expression for an element}
\newacronym[math=\elemeval]{elemeval}{Eq. eval}{Eigenvalues for an element}
\newacronym[math=c]{soundspeed}{Sound speed}{Speed of sound in m/s}
\newacronym[math=\eigenval]{eigenval}{eigenvalue}{A single eigenvalue}
\newacronym[math=\soundfraction]{soundfraction}{sound fraction}{Method parameter. Fraction of the speed of sound to use as smallest value for eigenvalues}
\newacronym[math=\smprecondsol]{smprecondsol}{sm precond sol}{Variable used as a solution to the smoother preconditioner problem}

\newacronym[]{dg}{DG}{Discontinous Galerkin}
\newacronym[]{fv}{FV}{Finite volume method/discretization}
\newacronym[]{mg}{MG}{Multigrid method}


\newglossarystyle{my}{%
  \setglossarystyle{long3colheader}%
  \renewcommand*{\glossaryheader}{%
    \toprule
    Abbr. & Symbol & Description \tabularnewline\midrule\endhead
    \bottomrule\endfoot
  }%
  \renewcommand*{\glsgroupskip}{%
     & &\tabularnewline}%
  \setlength\glsdescwidth{.08\textwidth}%
  \setlength\glspagelistwidth{.7\textwidth}%
  \renewcommand{\glossentry}[2]{%
    \glsentryitem{##1}\glstarget{##1}{\glossentryname{##1}} & $\acem{##1}$ & \glossentrydesc{##1}\tabularnewline
  }%
}

\setglossarystyle{my}


\title{GEF RAT-2 time stepping}

\begin{document}
\maketitle

\section{The Goal}

What we are trying to do is advance the system in time with the following formula:
\begin{equation}\label{eq:rat2_time_step}
    \vars_{n+1} = \vars_n + \systemmatlong{n}^{-1} \dt \rhssub{n}
    \text{,}
\end{equation}
where
\begin{equation}
    \jacobian_n = \left. \dfrac{\partial\rhs}{\partial\vars} \right|_{\vars_n}
\end{equation}

From eq.~\ref{eq:rat2_time_step}, we define the system matrix $\systemmat$ as
\begin{equation}\label{eq:system_mat}
    \systemmat_n := \systemmatlong{n}
\end{equation}

In order to do that, we need to solve a linear system
\begin{equation}\label{eq:system_to_solve}
    \systemmat_n \unknown_n = \rhssub{n}
\end{equation}
\emph{That's what we want to do!}

After that we can simply update the new state with
\begin{equation}
    \vars_{n+1} = \vars_n + \dt\unknown_n
\end{equation}

An outline of the steps we take to solve that problem of eq.~\ref{eq:system_to_solve}:
\begin{enumerate}
    \item Use \fgmres{} as an algorithm to solve the problem (with preconditioner) (sec.~\ref{sec:fgmres})
    \item Within \fgmres, call the preconditioner on every vector in the Krylov space 
    \item In the preconditioner, convert the discretization to FV (sec.~\ref{sec:finite_volume})
    \item Call the MG algorithm on the FV vector (sec.~\ref{sec:multigrid})
    \item Within MG, use a smoother on the input of each level. The smoother solves a pseudo time stepping problem (sec.~\ref{sec:smoother})
    \item Within the smoother, use a preconditioner on the pseudo time stepping problem
\end{enumerate}


\section{\fgmres}\label{sec:fgmres}

Soving for $x_n$ in Eq.~\ref{eq:system_to_solve} is complicated, so we use an iterative solver, \texttt{FGMRES}.
For that, we compute the basis of a Krylov space
$$K_{basis} = \left[\begin{matrix} \krylovstart & \systemmat_n\krylovstart & \systemmat_n^2 \krylovstart & \cdots \end{matrix}\right],$$
 starting with $\krylovstart$ as the residual given by our first guess
$\krylovstart = \rhssub{n} - \systemmat_n \vars_{n+1}^{guess}$, where the guess can be just $\bm{0}$.

Since the solver converges slowly with that particular set of vector, we precondition each of them
with $\gmrespremat^{-1}$, where $\gmrespremat\approx\systemmat$.
So we get 
\begin{equation}
\tilde{K}_{basis} = \left[\begin{matrix}
    \krylovstart_0 &
    \systemmat_n\gmrespremat^{-1}\krylovstart_0 &
    \systemmat_n\gmrespremat^{-1}\krylovstart_1 & \cdots \end{matrix}\right].
    \text{.}
\end{equation}


\section{Conversion to finite volume}\label{sec:finite_volume}

The \fgmres{} preconditioner starts by converting the input vector into a finite volume discretization. \emph{To be completed later.}

Once converted, we use the multigrid method on the resulting FV vector.

\textbf{Note:} We are using the weak form of the discontinous Galerkin method,
which at order 1 is equivalent to a FV method, using a Rusanov solver to reconstruct interface fluxes.


\section{Multigrid}\label{sec:multigrid}

\begin{algorithm}
    \caption{Multigrid}\label{alg:multigrid}
    \begin{algorithmic}[l]
        \State $L \gets $ number of grid levels
        \State  $\mgsol_L^0 \gets \bm{0}$ \Comment{Initial guess for $\mgsol_L$}
        \Procedure{MG}{$\mgin, \mgsol_l^0, l$} \Comment{On first call, $l = L$}
            \State $ \mgsol_l \gets \smoother(\systemmat_l, \mgin, \mgsol_l^0, \pseudodt_l) $
                    \Comment {Smoothe}
            \If{$l > 0$}
                \State $ \mgres_l \gets \restrict_l (\mgin_l - \systemmat_l\mgsol_l ) $
                        \Comment{Restrict residual}
                \State $ \mgsol_{l-1} \gets \bm{0} $ \Comment{Init lower level MG}
                \For{$i = 1..\gamma$}   \Comment{Using $\gamma = 1$ all the time}
                    \State $ \mgsol_{l-1} \gets MG(\mgres_l, \mgsol_{l-1}, l - 1) $ \Comment{$\mgres_l$ = ``$\mgin$'' for next grid level}
                \EndFor
                \State $ \mgsol_l \gets \mgsol_l + \prolong_l(\mgsol_{l-1}) $ \Comment{Prolong + apply correction}
            \EndIf
        \EndProcedure
    \end{algorithmic}
\end{algorithm}

\emph{For now when testing, we only perform these steps on the first level, so instead of multigrid, we're basically using the smoother $\smoother$ as a preconditioner}

\section{Smoother}\label{sec:smoother}

\subsection{Smoother problem}

The smoother solves a pseudo time stepping problem 
\begin{equation}\label{eq:smoother_pseudo_time_step}
    \dfrac{\partial\smvec}{\partial t^*} = \smootherrhs(\smvec)
    \text{.}
\end{equation}
We use
\begin{equation}
    \smootherrhs(\smvec) = \resrhs - \systemmat\smvec
    \text{,}
\end{equation}
where $\resrhs$ is the vector to which we are applying the MG preconditioner (when at the finest MG grid level) and $\smvec$ is the one we are smoothing.

\subsection{Implicit Runge-Kutta}

We do that by using a 3-stage implicit Runge-Kutta scheme:
\begin{subequations}
\begin{align}
    \smvec^{(0)}        &= \smvec_{n} \\
    \smvec^{(i)}        &= \smvec_{n} + \alpha_i\pseudodt\smprecondmat^{-1}\smootherrhs^{(i-1)}
            \label{eq:smoother_with_precond} \\
    \smvec_{n+1}        &= \smvec^{(3)} & \leftarrow \text{ the output}\\
    \smootherrhs^{(0)}  &= \smootherrhs(\smvec^{(0)}) \\
    \smootherrhs^{(i)}  &= \beta_i\smootherrhs(\smvec^{(i)}) + (1-\beta_i)\smootherrhs^{(i-1)}
\end{align}
\end{subequations}
where $\smprecondmat$ approximates the approximation $\awmat$ of the jacobian of the implicit system
\todo{that we should detail at eq.~\ref{eq:smoother_pseudo_time_step}}.

\begin{align}
    \smprecondmat &\approx \awmat
\end{align}
and
\begin{align}
    \awmat        &\approx \I + \eta\pseudodt\systemmat
\end{align}
where $\eta$ is a parameter.

\subsection{Smoother preconditioner}

We want to apply a preconditioner $\smprecondmat^{-1}$ to $\smootherrhs$
in eq.~\ref{eq:smoother_with_precond}. This gives rise to a problem
\begin{equation}
    \smprecondmat \smprecondsol = \smootherrhs \label{eq:precond_system}
\end{equation}

We define $\smprecondmat$ as one iteration of the symmetric Gauss-Seidel method:
\begin{align}
    \smprecondmat       &= (\diag + \uppermat) \diag^{-1} (\diag + \lowermat) \\
    \smprecondmat^{-1}  &= (\diag + \lowermat)^{-1} \diag (\diag + \uppermat)^{-1}
\end{align}
where $\lowermat$, $\diag$ and $\uppermat$ are respectively the lower triangular,
diagonal and upper triangular components of $\awmat$:
\begin{equation}
    \awmat = \lowermat + \diag + \uppermat
    \text{.}
\end{equation}
On a 2-dimensional structured grid, they are formed of $4 \times 4$ blocks:
\begin{subequations}\label{eq:lower_upper_diag_blocks}
\begin{align}
    \lowermat_{ij}^{(4\times4)} &= + \dfrac{\eta\pseudodt\dt}{2\Delta x^{ij}} \fluxjac^+_{n_{ij}} \label{eq:Lij} 
    = - \dfrac{\eta\pseudodt\dt}{2\Delta x^{ij}} \fluxjac^+_{n_{ji}}
    \\
    \uppermat_{ij}^{(4\times4)} &= + \dfrac{\eta\pseudodt\dt}{2\Delta x^{ij}} \fluxjac^-_{n_{ij}} \\
    \diag_{ij}^{(4\times4)} &= \I + \eta\pseudodt\I +
                    \dfrac{\eta\pseudodt\dt}{2}
                     \left[ \sum_{j \in N_i^+} \dfrac{\fluxjac^+_{n_i}}{\Delta x^{ij}}
                    - \sum_{j \in N_i^-} \dfrac{\fluxjac^-_{n_i}}{\Delta x^{ij}} \right] \label{eq:Dij}
                   \text{,}
\end{align}
\end{subequations}
where $\Delta x^{ij}$ is the width of an element along the axis from $i$ to $j$ and $\fluxjac_{n_{ij}}^+$ is the (positive) flux jacobian between these elements.

\subsubsection{Flux jacobian calculation}
For a certain element, we use
\begin{align}
    \fluxjac^+ &= \elemevecr \elemeval^+ \elemevecl \\
    \fluxjac^- &= \elemevecr \elemeval^- \elemevecl 
    \text{,}
\end{align}
where
\begin{equation}
    \elemjac = \elemevecr\elemeval\elemevecl
\end{equation}
(see appendix~\ref{app:flux_jacobian}).
The indices for the direction (either $x^1$ or $x^3$) of the flux are omitted here.
We define $\elemeval^+$ to be $\elemeval$ with all negative values set to $0$, and
$\elemeval^-$ to be $\elemeval$ with all positive values set to $0$. We also clamp each eigenvalue $\eigenval$ so that it's not too close to zero:
\begin{equation}
    |\eigenval| = \dfrac{1}{2}\left(\soundfraction\soundspeed + \dfrac{|\eigenval|^2}{\soundfraction\soundspeed}\right), |\eigenval| \le \soundfraction\soundspeed
\end{equation}
where $\soundspeed$ is the speed of sound and $\soundfraction$ is a fraction between $0$ and $1$
 \todo{(we set it to $0.5$ during tests)}.

For the flux between two elements $i$ and $j$, we use
\begin{equation}
    \fluxjac^+_{n_{ij}} = \dfrac{1}{2}\left(\fluxjac^+_{i,n_{ij}} + \fluxjac^+_{j,n_{ij}}\right)
    \text{,}
\end{equation}
where $\fluxjac^+_{i,n_{ij}}$ is the (positive) flux jacobian matrix for element $i$ along the $n_{ij}$ vector, which itself lies along either $x^1$ or $x^3$. So if $i$ and $j$ are neighbors along $x^1$ and $i$ comes before $j$ on that axis, we get
\begin{equation}
    \fluxjac^+_{n_{ij}} = \dfrac{1}{2}\left(\fluxjac^+_{i,(1)} + \fluxjac^+_{j,(1)}\right)
    \text{.}
\end{equation}
If, instead, $j$ comes before $i$ on that axis, we get
\begin{align}
    \fluxjac^+_{n_{ij}} &= -\dfrac{1}{2}\left(\fluxjac^+_{i,(1)} + \fluxjac^+_{j,(1)}\right) \\
                        &= -\fluxjac^+_{n_{ji}} \\
    \text{.}
\end{align}

\subsubsection{Solving the preconditioner system}

We can solve the preconditioner system of eq.~\ref{eq:precond_system} with a series of 
substitutions:
\begin{align}
    \smprecondmat \smprecondsol &= \smootherrhs \\
    (\diag + \uppermat) \diag^{-1} (\diag + \lowermat) \smprecondsol &= \smootherrhs
\end{align}

\begin{enumerate}
    \item $(\diag + \uppermat) \smprecondsol^{(a)} = \smootherrhs$ with backward substitution:
        \begin{equation}
            \smprecondsol^{(a)}_i =
             \diag^{-1}_{ii} \left(\smootherrhs_i - \sum_{j=i+1}^n \uppermat_{ij}\smprecondsol^{(a)}_j \right) \label{eq:backward_step}
        \end{equation}
    \item $\diag^{-1} \smprecondsol^{(b)} = \smprecondsol^{(a)}$ directly:
        \begin{equation}
            \smprecondsol^{(b)}_i = \diag_{ii}\smprecondsol^{(a)}_i
        \end{equation}
    \item $(\diag + \lowermat) \smprecondsol = \smprecondsol^{(b)}$ with forward substitution:
        \begin{equation}
            \smprecondsol_i = \diag^{-1}_{ii} \left(
                \smprecondsol^{(b)}_i - \sum_{j=0}^{i-1} \lowermat_{ij} \smprecondsol_j
                \right)
        \end{equation}
\end{enumerate}

Since $\lowermat_{ij}$ is only non-zero when elements $i$ and $j$ are neighbors,
eq.~\ref{eq:backward_step} can be reduced to
\begin{equation}
    \smprecondsol^{(a)}_i = \diag_{ii}^{-1} \left(
        \smootherrhs_i - \sum_{j\in N_i^+} \uppermat_{ij} \smprecondsol^{(a)}_j \right)
    \text{,}
\end{equation}
where $N_i^+$ is the set of neighbors of $i$ that are in the positive direction along each of the 2 axes (there are only 2 of them!).

\appendix

\section{Euler equations for the bubble}

\todo{To be completed}

Thermodynamics equation, for potential temperature $\theta$:
\begin{equation}
    \dfrac{ \partial \rho \theta}{\partial t } + \dfrac{ \rho \theta u }{\partial x } +  \dfrac{ \rho \theta w }{\partial z } = 0
\end{equation}


According to flux\_jacobian.pdf (from what I understand), we have a variable
$\vars = \left[\begin{matrix}
    \rho & \rho u^1 & \rho u^3 & \rho \theta
\end{matrix}\right]$
with the following equilibrium equations:
\begin{align}
    \dfrac{\partial \rho}{\partial t} + \dfrac{\partial \rho u^1}{\partial x^1} +
        \dfrac{\partial \rho u^3}{\partial x^3} &= 0 \\
    \dfrac{\partial \rho u^1}{\partial t} + \dfrac{\partial(\rho u^1u^1 + p)}{\partial x^1}
        + \dfrac{\partial \rho u^1u^3}{\partial x^3} &= 0 \\
    \dfrac{\partial \rho u^3}{\partial t} + \dfrac{\partial\rho u^1u^3}{\partial x^1}
        + \dfrac{\partial (\rho u^3u^3 + p)}{\partial x^3} &= \text{gravity stuff} \\
    \dfrac{\partial \rho \theta}{\partial t} +
        \dfrac{\partial \rho u^1 \theta}{\partial x^1} +
        \dfrac{\partial \rho u^3 \theta}{\partial x^3} &= \text{heat stuff}
\end{align}

In a more suitable form (for isolating the Jacobian):
\begin{align}
    \begin{matrix}
    \dfrac{\partial \rho}{\partial t} &=
        & - \dfrac{\partial \rho u^1}{\partial x^1}
        & - \dfrac{\partial \rho u^3}{\partial x^3} \\
    \dfrac{\partial \rho u^1}{\partial t} &=
        & - \dfrac{\partial(\rho u^1u^1 + p)}{\partial x^1}
        & - \dfrac{\partial \rho u^1u^3}{\partial x^3} \\
    \dfrac{\partial \rho u^3}{\partial t} &=
        & - \dfrac{\partial\rho u^1u^3}{\partial x^1}
        & - \dfrac{\partial(\rho u^3u^3 + p)}{\partial x^3} 
        & + \cdots \\
    \dfrac{\partial \rho \theta}{\partial t} &=
        & - \dfrac{\partial \rho u^1 \theta}{\partial x^1}
        & - \dfrac{\partial \rho u^3 \theta}{\partial x^3}
        & + \cdots
    \end{matrix}
\end{align}

\todo{Big question: It looks like the Jacobian for the $x^1$ direction would look like}
\begin{equation}
    \left( \begin{matrix}
        u^1 & 1 & 0 & 0 \\
        \soundspeed^2 - u^1u^1 & u^1 & 0 & 0 \\ 
        -u^1 u^3 & u^3 & u^1 & 0 \\
        -\theta u^1 & \theta & 0 & u^1
    \end{matrix} \right)
\end{equation}
rather than eq.~\ref{eq:elem_jac_1}. Also, not sure how to get a positive sign for $c^2$ (given that $p = \frac{\rho c^2}{\gamma}$ \todo{?})


\section{Flux jacobian for bubble equation}\label{app:flux_jacobian}

Variables are $\rho$, $u^1$ (horizontal velocity), $u^3$ (vertical velocity, axis where gravity is applied), $\theta$.

Along $x^1$:
\begin{align}
    \elemjac_{(1)} &= \left(
        \begin{matrix}
            0 & 1 & 0 & 0 \\
            \soundspeed^2 - u^1u^1 & 2u^1 & 0 & 0 \\ 
            -u^1 u^3 & u^3 & u^1 & 0 \\
            -\theta u^1 & \theta & 0 & u^1
        \end{matrix}
        \right) \label{eq:elem_jac_1}\\
    \elemeval_{(1)} &= \left(
        \begin{matrix}
            u^1 & 0 & 0 & 0 \\
            0 & u^1 & 0 & 0 \\
            0 & 0 & u^1 + \soundspeed & 0 \\
            0 & 0 & 0 & u^1 - \soundspeed
        \end{matrix}
        \right) \\
    \elemevecr_{(1)} &= \left(
        \begin{matrix}
            0 & 0 & \dfrac{1}{\theta} & \dfrac{1}{\theta} \\
            0 & 0 & \dfrac{u^1 + \soundspeed}{\theta} & \dfrac{u^1 - \soundspeed}{\theta} \\
            1 & 0 & \dfrac{u^3}{\theta} & \dfrac{u^3}{\theta} \\
            0 & 1 & 1 & 1
        \end{matrix}
        \right) \\
    \elemevecl_{(1)} &= \left(
        \begin{matrix}
            -u^3 & 0 & 1 & 0 \\
            -\theta & 0 & 0 & 1 \\
            -\dfrac{\theta(u^1 - \soundspeed)}{2\soundspeed} & \dfrac{\theta}{2\soundspeed} & 0 & 0 \\
            \dfrac{\theta(u^1 + \soundspeed)}{2\soundspeed} & -\dfrac{\theta}{2\soundspeed} & 0 & 0 
        \end{matrix}
        \right)
\end{align}

Along $x^3$:
\begin{align}
    \elemjac_{(3)} &= \left(
        \begin{matrix}
            0 & 0 & 1 & 0 \\
            -u^3u^1 & u^3 & u^1 & 0 \\ 
            \soundspeed^2 -u^3 u^3 & 0 & 2u^3 & 0 \\
            -\theta u^3 & 0 & \theta & u^3
        \end{matrix}
        \right) \\
    \elemeval_{(3)} &= \left(
        \begin{matrix}
            u^3 & 0 & 0 & 0 \\
            0 & u^3 & 0 & 0 \\
            0 & 0 & u^3 + \soundspeed & 0 \\
            0 & 0 & 0 & u^3 - \soundspeed
        \end{matrix}
        \right) \\
    \elemevecr_{(3)} &= \left(
        \begin{matrix}
            0 & 0 & \dfrac{1}{\theta} & \dfrac{1}{\theta} \\
            1 & 0 & \dfrac{u^1}{\theta} & \dfrac{u^1}{\theta} \\
            1 & 0 & \dfrac{u^3+\soundspeed}{\theta} & \dfrac{u^3-\soundspeed}{\theta} \\
            0 & 1 & 1 & 1
        \end{matrix}
        \right) \\
    \elemevecl_{(3)} &= \left(
        \begin{matrix}
            -u^1 & 1 & 0 & 0 \\
            -\theta & 0 & 0 & 1 \\
            -\dfrac{\theta(u^3 - \soundspeed)}{2\soundspeed} & 0 & \dfrac{\theta}{2\soundspeed} & 0 \\
            \dfrac{\theta(u^3 + \soundspeed)}{2\soundspeed} & 0 & -\dfrac{\theta}{2\soundspeed} & 0
        \end{matrix}
        \right)
\end{align}

\glsaddallunused
\printglossaries
\end{document}
