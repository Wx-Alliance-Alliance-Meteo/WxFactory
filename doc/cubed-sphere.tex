\documentclass{article}

\usepackage[colorlinks,bookmarksopen,bookmarksnumbered,citecolor=red,urlcolor=red]{hyperref}
\usepackage{moreverb}
\usepackage{amsmath}
\usepackage{amssymb}
\usepackage[margin=1.0in]{geometry}

\begin{document}

\title{The cubed-sphere in the four-dimensional tensor formalism}

\author{Martin Charron, St\'ephane Gaudreault \\
\\ \vspace{6pt} Recherche en pr\'evision num\'erique atmosph\'erique, 
\\ Environnement Canada, Dorval, Qc, Canada}

\maketitle

\begin{abstract}
The governing equations in the cubed-sphere coordinate system are written using the four-dimensional tensor formalism. The quartet deep/non-hydrostatic, shallow/non-hydrostatic, deep/hydrostatic, and shallow/hydrostatic under the spherical geopotential approximation is considered. These equations are also written in non-manifestly tensorial flux-form.
\end{abstract}

\section{Coordinate transformations}
In spherical coordinates, $\tilde{x}^0 = t$, $\tilde{x}^1 = \lambda$, $\tilde{x}^2 = \phi$, and $\tilde{x}^3 = r$, where $t$ is time, $\lambda$ longitude, $\phi$ latitude, and $r$ the distance from the origin of the coordinate system. In these coordinates, the covariant metric tensor is written
\begin{align}
\tilde{g}_{\mu \nu}=\left( \begin{array}{cccc}
1+(\Omega r\cos{\phi})^2 & \Omega(r\cos{\phi})^2 & 0 & 0 \\
\Omega(r\cos{\phi})^2  & (r\cos{\phi})^2   & 0 & 0 \\
0 & 0 & r^2 & 0 \\
0 & 0 & 0 & 1 \end{array} \right). \label{gconsp}
\end{align}
An infinitesimal space-time distance is writtem
\begin{align}
ds^2=\left( 1+(\Omega r \cos\phi)^2 \right) dt^2+2\Omega (r \cos\phi)^2 d\lambda dt + (r \cos\phi)^2 d\lambda^2 +r^2 d\phi^2 + dr^2.
\end{align}
Define $z=\epsilon^{-1} (r-a)$, where $a$ is the mean radius of the Earth, and $\epsilon$ is a non-dimensional parameter representing the aspect ratio of typical ``horizontal'' to ``vertical'' motion. Introduce a parameter $\Delta$, where $\Delta=1$ means that a deep atmosphere is considered, while $\Delta=0$ corresponds to a shallow atmosphere. In the coordinate $t$, $\lambda$, $\phi$, and $z$, the covariant metric tensor becomes
\begin{align}
\tilde{g}_{\mu \nu}=\left( \begin{array}{cccc}
1+(\Omega (a+\Delta\epsilon z)\cos{\phi})^2 & \Omega((a+\Delta\epsilon z)\cos{\phi})^2 & 0 & 0 \\
\Omega((a+\Delta\epsilon z)\cos{\phi})^2  & ((a+\Delta\epsilon z)\cos{\phi})^2   & 0 & 0 \\
0 & 0 & (a+\Delta\epsilon z)^2 & 0 \\
0 & 0 & 0 & \epsilon^2 \end{array} \right). \label{gconsp}
\end{align}

Consider panel $1$ of the cubed-sphere coordinates $S$. Define time, the angles $x^1$ and $x^2$, and the radial coordinate $x^3$ as
\begin{align}
x^0&=t, \\
x^{1}&=\lambda, \\
x^{2}&=\tan^{-1} \left( \frac{\tan \phi}{\cos \lambda}\right), \\
x^{3}&=z.
\end{align}
Define $X=\tan x^1$, $Y=\tan x^2$, and $\delta^2=1+X^2+Y^2$. This leads to the following relations:
\begin{align}
\cos^2 \lambda&=\frac{1}{1+X^2}, \\
\cos^2 \phi&=\frac{1+X^2}{\delta^2}, \\
d\lambda&=dx^1, \\
d\phi^2&=\frac{1+X^2}{\delta^2} \left( X^2Y^2 dx^1 dx^1 -2XY(1+Y^2)dx^1dx^2 +(1+Y^2)^2dx^2dx^2  \right), \\
\frac{\partial x^2}{\partial \lambda}&=\frac{XY}{1+Y^2}, \\
\left( \frac{\partial x^2}{\partial \phi}\right)^2&=1+X^2.
\end{align}
The non-zero components of the covariant metric tensor in the coordinate system $S_1$ (i.e., panel $1$) become
\begin{align}
g_{00}&=1+(1+X^2) \left( \frac{(a+\Delta\epsilon z)\Omega}{\delta} \right)^2, \\
g_{01}&=(1+X^2) \Omega \left( \frac{a+\Delta\epsilon z}{\delta} \right)^2 =g_{10}, \\
g_{11}&=(1+Y^2) \left( \frac{(a+\Delta\epsilon z)(1+X^2)}{\delta^2} \right)^2, \\
g_{12}&=-XY(1+X^2)(1+Y^2) \left( \frac{a+\Delta\epsilon z}{\delta^2} \right)^2 = g_{21}, \\
g_{22}&=(1+X^2) \left( \frac{(a+\Delta\epsilon z)(1+Y^2)}{\delta^2} \right)^2, \\
g_{33}&=\epsilon^2.
\end{align}
The square root of the determinant of the covariant metric tensor is
\begin{align}
\sqrt{g}=\frac{\epsilon (a+\Delta\epsilon z)^2 (1+X^2)(1+Y^2)}{\delta^3}.
\end{align}
Its derivatives are
\begin{align}
(\sqrt{g})_{,1}&=\frac{\epsilon (a+\Delta\epsilon z)^2 X(1+X^2)(1+Y^2)(2Y^2-X^2-1)}{\delta^5}, \\
(\sqrt{g})_{,2}&=\frac{\epsilon (a+\Delta\epsilon z)^2 Y(1+X^2)(1+Y^2)(2X^2-Y^2-1)}{\delta^5}, \\
(\sqrt{g})_{,3}&=\frac{2\Delta \epsilon^2 (a+\Delta\epsilon z) (1+X^2)(1+Y^2)}{\delta^3}.
\end{align}
From $g_{\mu \alpha}g^{\mu \beta}=\delta^\beta_\alpha$, one calculates the contravariant metric tensor. Its non-zero components are
\begin{align}
g^{00}&=1, \\
g^{01}&=-\Omega =g^{10}, \\
g^{02}&=\frac{-\Omega XY}{1+Y^2} = g^{20}, \\
g^{11}&=\Omega^2 + \frac{\delta^2}{(a+\Delta \epsilon z)^2(1+X^2)}, \\
g^{12}&=\frac{XY}{1+Y^2}\left(\Omega^2+\frac{\delta^2}{(a+\Delta \epsilon z)^2(1+X^2)} \right)=g^{21}, \\
g^{22}&=\left( \frac{\Omega XY}{1+Y^2} \right)^2 + \frac{\delta^2}{(a+\Delta \epsilon z)^2(1+Y^2)}, \\
g^{33}&=\epsilon^{-2}.
\end{align}
The non-zero components of the tensor $h^{\mu \nu}=g^{\mu \nu}-g^{0 \mu}g^{0 \nu}$ are
\begin{align}
h^{11}&=\frac{\delta^2}{(a+\Delta \epsilon z)^2(1+X^2)}, \\
h^{12}&=\frac{XY\delta^2}{(a+\Delta \epsilon z)^2(1+X^2)(1+Y^2)}=h^{21}, \\
h^{22}&=\frac{\delta^2}{(a+\Delta \epsilon z)^2(1+Y^2)}, \\
h^{33}&=\epsilon^{-2}.
\end{align}
These tensors have the same form for panels $p=1,2,3,4$ since
\begin{align}
X&=\tan x^1 = \tan \left( \lambda -(p-1)\frac{\pi}{2} \right), \\
Y&=\tan x^2 = \frac{\tan \phi}{\cos \left( \lambda -(p-1)\frac{\pi}{2}\right)}.
\end{align}

The panel covering the Arctic region is called panel $5$. In the cubed-sphere coordinate system $S_5$, $x^1$ and $x^2$ are related to longitude $\lambda$ and latitude $\phi$ by
\begin{align}
x^1&=\tan^{-1} \left( \frac{\sin \lambda}{\tan \phi}  \right), \\
x^2&=- \tan^{-1} \left( \frac{\cos \lambda}{\tan \phi}  \right).
\end{align}
Define again $X=\tan x^1$, $Y=\tan x^2$, and $\delta^2=1+X^2+Y^2$. The non-zero components of the covariant metric tensor in $S_5$ are
\begin{align}
g_{00}&=1+(X^2+Y^2) \left( \frac{(a+\Delta\epsilon z)\Omega}{\delta} \right)^2, \\
g_{01}&=-Y(1+X^2) \Omega \left( \frac{a+\Delta\epsilon z}{\delta} \right)^2=g_{10}, \\
g_{02}&=X(1+Y^2) \Omega \left( \frac{a+\Delta\epsilon z}{\delta} \right)^2=g_{20}, \\
g_{11}&=(1+Y^2) \left( \frac{(a+\Delta\epsilon z)(1+X^2)}{\delta^2} \right)^2, \\
g_{12}&=-XY(1+X^2)(1+Y^2) \left( \frac{a+\Delta\epsilon z}{\delta^2} \right)^2=g_{21}, \\
g_{22}&=(1+X^2) \left( \frac{(a+\Delta\epsilon z)(1+Y^2)}{\delta^2} \right)^2, \\
g_{33}&=\epsilon^2.
\end{align}
From $g_{\mu \alpha}g^{\mu \beta}=\delta^\beta_\alpha$, one calculates the contravariant metric tensor in $S_5$. Its non-zero components are
\begin{align}
g^{00}&=1, \\
g^{01}&=\frac{\Omega Y}{1+X^2}=g^{10}, \\
g^{02}&=\frac{-\Omega X}{1+Y^2} = g^{20}, \\
g^{11}&=\left( \frac{\Omega Y}{1+X^2}\right)^2+  \frac{\delta^2}{(a+\Delta \epsilon z)^2(1+X^2)}, \\
g^{12}&=\frac{-\Omega^2 XY}{(1+X^2)(1+Y^2)}+\frac{XY\delta^2}{(a+\Delta \epsilon z)^2(1+X^2)(1+Y^2)}=g^{21}, \\
g^{22}&=\left( \frac{\Omega X}{1+Y^2} \right)^2+\frac{\delta^2}{(a+\Delta \epsilon z)^2(1+Y^2)}, \\
g^{33}&=\epsilon^{-2}.
\end{align}
The non-zero components of the tensor $h^{\mu \nu}=g^{\mu \nu}-g^{0 \mu}g^{0 \nu}$ remain
\begin{align}
h^{11}&=\frac{\delta^2}{(a+\Delta \epsilon z)^2(1+X^2)}, \\
h^{12}&=\frac{XY\delta^2}{(a+\Delta \epsilon z)^2(1+X^2)(1+Y^2)}=h^{21}, \\
h^{22}&=\frac{\delta^2}{(a+\Delta \epsilon z)^2(1+Y^2)}, \\
h^{33}&=\epsilon^{-2}.
\end{align}

The panel covering Antarctica is called panel $6$. In the cubed-sphere coordinate system $S_6$, $x^1$ and $x^2$ are related to longitude $\lambda$ and latitude $\phi$ by
\begin{align}
x^1&=- \tan^{-1} \left( \frac{\sin \lambda}{\tan \phi}  \right), \\
x^2&=- \tan^{-1} \left( \frac{\cos \lambda}{\tan \phi}  \right).
\end{align}
Define again $X=\tan x^1$, $Y=\tan x^2$, and $\delta^2=1+X^2+Y^2$. The non-zero components of the covariant metric tensor in $S_6$ are
\begin{align}
g_{00}&=1+(X^2+Y^2) \left( \frac{(a+\Delta\epsilon z)\Omega}{\delta} \right)^2, \\
g_{01}&=Y(1+X^2) \Omega \left( \frac{a+\Delta\epsilon z}{\delta} \right)^2=g_{10}, \\
g_{02}&=-X(1+Y^2) \Omega \left( \frac{a+\Delta\epsilon z}{\delta} \right)^2=g_{20}, \\
g_{11}&=(1+Y^2) \left( \frac{(a+\Delta\epsilon z)(1+X^2)}{\delta^2} \right)^2, \\
g_{12}&=-XY(1+X^2)(1+Y^2) \left( \frac{a+\Delta\epsilon z}{\delta^2} \right)^2=g_{21}, \\
g_{22}&=(1+X^2) \left( \frac{(a+\Delta\epsilon z)(1+Y^2)}{\delta^2} \right)^2, \\
g_{33}&=\epsilon^2.
\end{align}
From $g_{\mu \alpha}g^{\mu \beta}=\delta^\beta_\alpha$, one calculates the contravariant metric tensor in $S_6$. Its non-zero components are
\begin{align}
g^{00}&=1, \\
g^{01}&=\frac{-\Omega Y}{1+X^2}=g^{10}, \\
g^{02}&=\frac{\Omega X}{1+Y^2} = g^{20}, \\
g^{11}&=\left( \frac{\Omega Y}{1+X^2}\right)^2+  \frac{\delta^2}{(a+\Delta \epsilon z)^2(1+X^2)}, \\
g^{12}&=\frac{-\Omega^2 XY}{(1+X^2)(1+Y^2)}+\frac{XY\delta^2}{(a+\Delta \epsilon z)^2(1+X^2)(1+Y^2)}=g^{21}, \\
g^{22}&=\left( \frac{\Omega X}{1+Y^2} \right)^2+\frac{\delta^2}{(a+\Delta \epsilon z)^2(1+Y^2)}, \\
g^{33}&=\epsilon^{-2}.
\end{align}
The non-zero components of the tensor $h^{\mu \nu}=g^{\mu \nu}-g^{0 \mu}g^{0 \nu}$ remain
\begin{align}
h^{11}&=\frac{\delta^2}{(a+\Delta \epsilon z)^2(1+X^2)}, \\
h^{12}&=\frac{XY\delta^2}{(a+\Delta \epsilon z)^2(1+X^2)(1+Y^2)}=h^{21}, \\
h^{22}&=\frac{\delta^2}{(a+\Delta \epsilon z)^2(1+Y^2)}, \\
h^{33}&=\epsilon^{-2}.
\end{align}

In tensor notation, the continuity ($\mu=0$) and momentum ($\mu=i$) equations are
\begin{align}
{T^{\mu \nu}}_{:\nu}=-\rho h^{\mu \nu} \Phi_{,\nu}, \label{mme}
\end{align}
where
\begin{align}
T^{\mu \nu}&=\rho u^{\mu} u^{\nu} + h^{\mu \nu} p + \sigma^{\mu \nu}, \\
u^{\mu}&=\frac{dx^\mu}{dt}, \\
\sigma^{\mu \nu}&=\text{viscous effects}, \\
\Phi&=\text{gravitational potential}.
\end{align}
The continuity ($\mu=0$) equation is $(\rho u^{\nu})_{:\nu}=0$. After multiplying by $\sqrt{g}$, it becomes
\begin{align}
\sqrt{g} (\rho u^{\nu})_{:\nu} = (\sqrt{g} \rho u^{\nu})_{,\nu} = 0,
\end{align}
i.e. a flux form. Contract Eq.\ \eqref{mme} with $g_{\alpha \mu}$ and multiply by $\sqrt{g}$. It leads to
\begin{align}
\sqrt{g} {{T_\alpha}^\nu}_{:\nu}=- \sqrt{g} \rho {h_\alpha}^\nu \Phi_{,\nu} = - \sqrt{g} \rho \Phi_{,\alpha} + \sqrt{g} \rho \delta_\alpha^0 \Phi^{,0} \label{momc}
\end{align}
Write
\begin{align}
\sqrt{g} {{T_\alpha}^\nu}_{:\nu}=\sqrt{g} \left( {{T_\alpha}^\nu}_{,\nu} + \Gamma^\nu_{\beta\nu} {T_\alpha}^\beta - \Gamma^\beta_{\alpha\nu}{T_{\beta}}^\nu  \right). \label{divt}
\end{align}
Use
\begin{align}
\sqrt{g} \Gamma^\nu_{\beta\nu} &= (\sqrt{g})_{,\beta}, \\
\Gamma^\beta_{\alpha\nu}{T_{\beta}}^\nu &= \Gamma_{\beta\alpha\nu}T^{\beta\nu}=\frac{1}{2}\left( \Gamma_{\beta\alpha\nu} + \Gamma_{\nu\alpha\beta} \right)T^{\beta\nu}=\frac{1}{2}T^{\beta\nu} g_{\beta\nu , \alpha},
\end{align}
where $\Gamma_{\beta\alpha\nu} + \Gamma_{\nu\alpha\beta}=g_{\beta\nu , \alpha}$ has been used. Eq.\ \eqref{divt} becomes
\begin{align}
\sqrt{g} {{T_\alpha}^\nu}_{:\nu}=\left( \sqrt{g}{{T_\alpha}^\nu}\right)_{,\nu} - \frac{1}{2} \sqrt{g} T^{\beta\nu} g_{\beta\nu , \alpha}.
\end{align}
Consequently, Eq.\ \eqref{momc} can be rewritten
\begin{align}
\left( \sqrt{g}{{T_\alpha}^\nu}\right)_{,\nu} = - \sqrt{g} \rho \Phi_{,\alpha} + \sqrt{g} \rho \delta_\alpha^0 \Phi^{,0} + \frac{1}{2} \sqrt{g} T^{\beta\nu} g_{\beta\nu , \alpha}.
\end{align}
If one chooses $\alpha=i$, where $i=1,2,3$, the preceding equation becomes
\begin{align}
\left( \sqrt{g}{{T_i}^\nu}\right)_{,\nu} = - \sqrt{g} \rho \Phi_{,i} + \frac{1}{2} \sqrt{g} T^{\beta\nu} g_{\beta\nu , i}.
\end{align}
This is a flux form expression of the momentum equations. Using $T^{00}=\rho$ and $T^{0j}=T^{j0}=\rho u^j$, it can be further expanded as
\begin{align}
\left( \sqrt{g}{{T_i}^\nu}\right)_{,\nu} = - \sqrt{g} \rho \left( \Phi - \frac{1}{2} g_{00} \right)_{,i} + \sqrt{g} \rho u^j g_{0j , i} + \frac{1}{2} \sqrt{g} T^{jk} g_{jk , i}.
\end{align}
Note that
\begin{align}
\Phi - \frac{1}{2} g_{00} \equiv \Phi_e
\end{align}
is the effective gravitational potential, also called geopotential. In geopotential coordinates, one has
\begin{align}
\left( \Phi - \frac{1}{2} g_{00} \right)_{,1}=0=\left( \Phi - \frac{1}{2} g_{00} \right)_{,2}.
\end{align}
One defines the gravitational acceleration $G$ as
\begin{align}
\left( \Phi - \frac{1}{2} g_{00} \right)_{,3}\equiv \epsilon G.
\end{align}
Note also that
\begin{align}
\left( \sqrt{g}{{T_i}^\nu}\right)_{,\nu} = \frac{\partial(\sqrt{g} \rho u_i)}{\partial t} + \frac{\partial (\sqrt{g} [\rho u_iu^j+\delta^j_ip+{\sigma_i}^j])}{\partial x^j}.
\end{align}
After some algebra, the flux form of the inviscid momentum equation may be rewritten
\begin{align}
\left( \sqrt{g}{T_i}^\nu\right)_{,\nu} = - \sqrt{g} \rho \left( \Phi - \frac{1}{2} g_{00} \right)_{,i} + p (\sqrt{g})_{,i}+ \sqrt{g} \rho u^j g_{0j , i} + \frac{1}{2} \sqrt{g} \rho u^ju^k g_{jk , i},
\end{align}
or
\begin{align}
\left( \sqrt{g}\rho u_iu^\nu\right)_{,\nu} = - \sqrt{g} \rho \left( \Phi - \frac{1}{2} g_{00} \right)_{,i} - \sqrt{g}p_{,i}+ \sqrt{g} \rho u^j g_{0j , i} + \frac{1}{2} \sqrt{g} \rho u^ju^k g_{jk , i}.
\end{align}
The continuity equation $(\sqrt{g}\rho u^\nu)_{,\nu}=0$ may be used to derive the inviscid Lagrangian form of the covariant 4-vector 
\begin{align}
\frac{du_i}{dt} = - \left( \Phi - \frac{1}{2} g_{00} \right)_{,i} - \frac{1}{\rho}p_{,i}+ u^j g_{0j , i} + \frac{1}{2} u^ju^k g_{jk , i},
\end{align}
where
\begin{align}
\frac{du_i}{dt} = u^\nu u_{i,\nu}.
\end{align}
The covariant components $u_i$ are related to the contravariant components through
\begin{align}
u^k=g^{k\mu}u_\mu=(h^{k\mu}+g^{0k}g^{0\mu})u_\mu=h^{k\mu}u_\mu+g^{0k}u^0=h^{ki}u_i+g^{0k}.
\end{align}

An alternative flux form momentum equation may be obtained using contravariant vectors. After some algebra, one gets
\begin{align}
\left( \sqrt{g}\rho u^iu^\nu\right)_{,\nu} = - \sqrt{g} \rho h^{ij} g_{j0,0} - \sqrt{g} \rho h^{ij}\left( \Phi - \frac{1}{2} g_{00} \right)_{,j} - \sqrt{g}h^{ij}p_{,j} -2\sqrt{g} \rho \Gamma^i_{j0} u^j - \sqrt{g} \rho \Gamma^i_{jk}u^ju^k,
\end{align}
where
\begin{align}
\Gamma^i_{\mu\nu}=\frac{1}{2} g^{i\beta}(g_{\beta\mu,\nu}+g_{\beta\nu,\mu}-g_{\mu\nu,\beta})=\frac{1}{2} h^{ij}(g_{j\mu,\nu}+g_{j\nu,\mu}-g_{\mu\nu,j}).
\end{align}
The term $g_{j0,0}$ vanishes when a static coordinate system is employed.

Another flux form momentum equation is obtained from the the following considerations. Starting from
\begin{align}
{{T_\mu}^\nu}_{:\nu} = -\rho \Phi_{,\mu} + \delta^0_\mu \rho \Phi^{,0}, \label{li}
\end{align}
one writes
\begin{align}
\frac{Du_\mu}{Dt}= -\Phi_{,\mu} -\frac{1}{\rho} p_{,\mu} + \delta^0_\mu \left( \Phi^{,0} + \frac{1}{\rho} p^{,0}\right).
\end{align}
The left-hand side is rewritten
\begin{align}
\frac{Du_\mu}{Dt}&=u^\nu u_{\mu:\nu} + u^\nu u_{\nu:\mu}-u^\nu u_{\nu:\mu}, \\
                 &= \left( \frac{1}{2} u^\nu u_{\nu}\right)_{,\mu}-u^\nu (u_{\nu:\mu} -u_{\mu:\nu} ), \\
		 &= \left( \frac{1}{2} u^\nu u_{\nu}\right)_{,\mu}-u^\nu (u_{\nu,\mu} -u_{\mu,\nu} ), \\
		 &= \frac{\partial u_\mu}{\partial t} + \left( \frac{1}{2} u^\nu u_{\nu} -u_0\right)_{,\mu} -u^j (u_{j,\mu} -u_{\mu,j} ), \\
		 &= \frac{\partial u_\mu}{\partial t} + \left( \frac{1}{2} u^j u_j -\frac{1}{2}u_0\right)_{,\mu} -u^j (u_{j,\mu} -u_{\mu,j} ).
\end{align}
Choosing $\mu=i$,
\begin{align}
\frac{\partial u_i}{\partial t} + \left( \frac{1}{2} u^j u_j +\Phi -\frac{1}{2}u_0\right)_{,i} =u^j (u_{j,i} -u_{i,j} ) -\frac{1}{\rho} p_{,i}.
\end{align}
Define the geopotential $\Phi_e=\Phi-\frac{1}{2}g_{00}$ and write $u_0=1-g^{0j}u_j$. The previous equation becomes
\begin{align}
\frac{\partial u_i}{\partial t} + \left( \frac{1}{2} u^j u_j +\Phi_e +\frac{1}{2}g^{0j}u_j+\frac{1}{2}g_{00}\right)_{,i} =u^j (u_{j,i} -u_{i,j} ) -\frac{1}{\rho} p_{,i},
\end{align}
or
\begin{align}
\frac{\partial u_i}{\partial t} + \left( \frac{1}{2} h^{jk}u_j u_k +\Phi_e + g^{0j}u_j+\frac{1}{2}g_{00}\right)_{,i} =u^j (u_{j,i} -u_{i,j} ) -\frac{1}{\rho} p_{,i},
\end{align}
or
\begin{align}
\frac{\partial u_i}{\partial t} + \left( \frac{1}{2} g_{jk}u^j u^k +\Phi_e \right)_{,i} =u^j (u_{j,i} -u_{i,j} ) -\frac{1}{\rho} p_{,i}.
\end{align}
In this flux form, the Christoffel symbols do not appear explicitly.

Since derivarives of covariant metric tensors may be needed, they are given below. For all six panels, one has
\begin{align}
g_{11,1}&=\frac{4XY^2(1+X^2)^2(1+Y^2)(a+\Delta \epsilon z)^2}{\delta^6}, \\
g_{11,2}&=\frac{2Y(1+X^2)^2(1+Y^2)(X^2-Y^2-1)(a+\Delta \epsilon z)^2}{\delta^6}, \\
g_{11,3}&=\frac{2\Delta \epsilon(1+X^2)^2(1+Y^2)(a+\Delta \epsilon z)}{\delta^4}, \\
g_{12,1}&=\frac{-Y(1+X^2)(1+Y^2)(X^2[3Y^2-X^2]+1+Y^2)(a+\Delta \epsilon z)^2}{\delta^6}=g_{21,1}, \\
g_{12,2}&=\frac{-X(1+X^2)(1+Y^2)(Y^2[3X^2-Y^2]+1+X^2)(a+\Delta \epsilon z)^2}{\delta^6}=g_{21,2}, \\
g_{12,3}&=\frac{-2\Delta \epsilon XY(1+X^2)(1+Y^2)(a+\Delta \epsilon z)}{\delta^4}=g_{21,3}, \\
g_{22,1}&=\frac{2X(1+X^2)(1+Y^2)^2(Y^2-X^2-1)(a+\Delta \epsilon z)^2}{\delta^6}, \\
g_{22,2}&=\frac{4X^2Y(1+X^2)(1+Y^2)^2(a+\Delta \epsilon z)^2}{\delta^6}, \\
g_{22,3}&=\frac{2\Delta \epsilon(1+X^2)(1+Y^2)^2(a+\Delta \epsilon z)}{\delta^4}, \\
g_{33,1}&=g_{33,2}=g_{33,3}=0.
\end{align}
For panels $1$ to $4$, one has
\begin{align}
g_{01,1}&=\frac{2\Omega XY^2(1+X^2)(a+\Delta \epsilon z)^2}{\delta^4}=g_{10,1}, \\
g_{01,2}&=\frac{-2\Omega Y(1+X^2)(1+Y^2)(a+\Delta \epsilon z)^2}{\delta^4}=g_{10,2}, \\
g_{01,3}&=\frac{2\Omega \Delta \epsilon (1+X^2)(a+\Delta \epsilon z)}{\delta^2}=g_{10,3}.
\end{align}
For panel $5$, one has
\begin{align}
g_{01,1}&=\frac{-2\Omega XY^3(1+X^2)(a+\Delta \epsilon z)^2}{\delta^4}=g_{10,1}, \\
g_{01,2}&=\frac{-\Omega (1+X^2)(1+Y^2)(1+X^2-Y^2)(a+\Delta \epsilon z)^2}{\delta^4}=g_{10,2}, \\
g_{01,3}&=\frac{-2\Omega \Delta \epsilon Y(1+X^2)(a+\Delta \epsilon z)}{\delta^2}=g_{10,3}, \\
g_{02,1}&=\frac{\Omega (1+X^2)(1+Y^2)(1-X^2+Y^2)(a+\Delta \epsilon z)^2}{\delta^4}=g_{20,1}, \\
g_{02,2}&=\frac{2\Omega X^3Y(1+Y^2)(a+\Delta \epsilon z)^2}{\delta^4}=g_{20,2}, \\
g_{02,3}&=\frac{2\Omega \Delta \epsilon X(1+Y^2)(a+\Delta \epsilon z)}{\delta^2}=g_{20,3}.
\end{align}
For panel $6$, one has
\begin{align}
g_{01,1}&=\frac{2\Omega XY^3(1+X^2)(a+\Delta \epsilon z)^2}{\delta^4}=g_{10,1}, \\
g_{01,2}&=\frac{\Omega (1+X^2)(1+Y^2)(1+X^2-Y^2)(a+\Delta \epsilon z)^2}{\delta^4}=g_{10,2}, \\
g_{01,3}&=\frac{2\Omega \Delta \epsilon Y(1+X^2)(a+\Delta \epsilon z)}{\delta^2}=g_{10,3}, \\
g_{02,1}&=\frac{-\Omega (1+X^2)(1+Y^2)(1-X^2+Y^2)(a+\Delta \epsilon z)^2}{\delta^4}=g_{20,1}, \\
g_{02,2}&=\frac{-2\Omega X^3Y(1+Y^2)(a+\Delta \epsilon z)^2}{\delta^4}=g_{20,2}, \\
g_{02,3}&=\frac{-2\Omega \Delta \epsilon X(1+Y^2)(a+\Delta \epsilon z)}{\delta^2}=g_{20,3}.
\end{align}

Consider now a time-independent, but otherwise general, radial coordinate in $\tilde{S}$. The coordinate transformation is
\begin{align}
\tilde{x}^0&=x^0, \\
\tilde{x}^1&=x^1, \\
\tilde{x}^2&=x^2, \\
\tilde{x}^3&=\eta(x^1,x^2,x^3).
\end{align}
These relations may be inverted:
\begin{align}
x^0&=\tilde{x}^0, \\
x^1&=\tilde{x}^1, \\
x^2&=\tilde{x}^2, \\
x^3&=z(\tilde{x}^1,\tilde{x}^2,\tilde{x}^3).
\end{align}
From these coordinate transformations, one calculates the metric tensors in $\tilde{S}$. The non-zero components are
\begin{align}
\tilde{g}_{00}&=g_{00}, \\
\tilde{g}_{01}&=g_{01}, \\
\tilde{g}_{02}&=g_{02}, \\
\tilde{g}_{11}&=g_{11} + \epsilon^2 \left( \frac{\partial z}{\partial \tilde{x}^1} \right)^2, \\
\tilde{g}_{12}&=g_{12} + \epsilon^2 \frac{\partial z}{\partial \tilde{x}^1} \frac{\partial z}{\partial \tilde{x}^2}, \\
\tilde{g}_{13}&=\epsilon^2 \frac{\partial z}{\partial \tilde{x}^1} \frac{\partial z}{\partial \eta}, \\
\tilde{g}_{22}&=g_{22} + \epsilon^2 \left( \frac{\partial z}{\partial \tilde{x}^2} \right)^2, \\
\tilde{g}_{23}&=\epsilon^2 \frac{\partial z}{\partial \tilde{x}^2} \frac{\partial z}{\partial \eta}, \\
\tilde{g}_{33}&=\epsilon^2 \left( \frac{\partial z}{\partial \eta} \right)^2, \\
\tilde{g}^{00}&=1, \\
\tilde{g}^{01}&=g^{01}, \\
\tilde{g}^{02}&=g^{02}, \\
\tilde{g}^{03}&=-\left( \frac{\partial z}{\partial \eta} \right)^{-1} \left( \frac{\partial z}{\partial \tilde{x}^1} g^{01} + \frac{\partial z}{\partial \tilde{x}^2} g^{02} \right), \\
\tilde{g}^{11}&=g^{11}, \\
\tilde{g}^{12}&=g^{12}, \\
\tilde{g}^{13}&=-\left( \frac{\partial z}{\partial \eta} \right)^{-1} \left( \frac{\partial z}{\partial \tilde{x}^1} g^{11} + \frac{\partial z}{\partial \tilde{x}^2} g^{12} \right), \\
\tilde{g}^{22}&=g^{22}, \\
\tilde{g}^{23}&=-\left( \frac{\partial z}{\partial \eta} \right)^{-1} \left( \frac{\partial z}{\partial \tilde{x}^1} g^{12} + \frac{\partial z}{\partial \tilde{x}^2} g^{22} \right), \\
\tilde{g}^{33}&=\left( \frac{\partial z}{\partial \eta} \right)^{-2} \left[ \epsilon^{-2} + \left( \frac{\partial z}{\partial \tilde{x}^1} \right)^2 g^{11} + \left( \frac{\partial z}{\partial \tilde{x}^2} \right)^2 g^{22} +2\frac{\partial z}{\partial \tilde{x}^1} \frac{\partial z}{\partial \tilde{x}^2} g^{12}\right], \\
\tilde{h}^{11}&=h^{11}, \\
\tilde{h}^{12}&=h^{12}, \\
\tilde{h}^{13}&=-\left( \frac{\partial z}{\partial \eta} \right)^{-1} \left( \frac{\partial z}{\partial \tilde{x}^1} h^{11} + \frac{\partial z}{\partial \tilde{x}^2} h^{12} \right), \\
\tilde{h}^{22}&=h^{22}, \\
\tilde{h}^{23}&=-\left( \frac{\partial z}{\partial \eta} \right)^{-1} \left( \frac{\partial z}{\partial \tilde{x}^1} h^{12} + \frac{\partial z}{\partial \tilde{x}^2} h^{22} \right), \\
\tilde{h}^{33}&=\left( \frac{\partial z}{\partial \eta} \right)^{-2} \left[ \epsilon^{-2} + \left( \frac{\partial z}{\partial \tilde{x}^1} \right)^2 h^{11} + \left( \frac{\partial z}{\partial \tilde{x}^2} \right)^2 h^{22} +2\frac{\partial z}{\partial \tilde{x}^1} \frac{\partial z}{\partial \tilde{x}^2} h^{12}\right].
\end{align}
One may then find
\begin{align}
\sqrt{\tilde{g}}=\frac{\epsilon(a+\Delta\epsilon z)^2 (1+X^2)(1+Y^2)}{\delta^3}\left|\frac{\partial z}{\partial\eta}\right|. \label{rgt}
\end{align}

In $\tilde{S}$, the derivatives of the covariant metric tensor are given below. For all 6 panels, the terms with components $i$ and $j$ are
\begin{align}
\tilde{g}_{11,1}&=\frac{4XY^2(1+X^2)^2(1+Y^2)(a+\Delta \epsilon z)^2}{\delta^6} + 2\epsilon \frac{\partial z}{\partial \tilde{x}^1} \left[ \frac{\Delta (1+X^2)^2(1+Y^2)(a+\Delta \epsilon z)}{\delta^4} + \epsilon \frac{\partial^2 z}{\partial \tilde{x}^1 \partial \tilde{x}^1}  \right], \\
\tilde{g}_{11,2}&=\frac{2Y(1+X^2)^2(1+Y^2)(X^2-Y^2-1)(a+\Delta \epsilon z)^2}{\delta^6}+\frac{2\Delta \epsilon(1+X^2)^2(1+Y^2)(a+\Delta \epsilon z)}{\delta^4} \frac{\partial z}{\partial \tilde{x}^2} \nonumber \\ &+ 2\epsilon^2 \frac{\partial z}{\partial \tilde{x}^1}\frac{\partial^2 z}{\partial \tilde{x}^1 \partial \tilde{x}^2}, \\
\tilde{g}_{11,3}&=\frac{2\Delta \epsilon(1+X^2)^2(1+Y^2)(a+\Delta \epsilon z)}{\delta^4} \frac{\partial z}{\partial \eta} + 2\epsilon^2 \frac{\partial z}{\partial \tilde{x}^1}\frac{\partial^2 z}{\partial \tilde{x}^1 \partial \eta}, \\
\tilde{g}_{12,1}&=\frac{-Y(1+X^2)(1+Y^2)(X^2[3Y^2-X^2]+1+Y^2)(a+\Delta \epsilon z)^2}{\delta^6} \nonumber \\ &- \frac{2\Delta \epsilon XY(1+X^2)(1+Y^2)(a+\Delta \epsilon z)}{\delta^4} \frac{\partial z}{\partial \tilde{x}^1} + \epsilon^2 \left[ \frac{\partial z}{\partial \tilde{x}^2} \frac{\partial^2 z}{\partial \tilde{x}^1\partial \tilde{x}^1} + \frac{\partial z}{\partial \tilde{x}^1} \frac{\partial^2 z}{\partial \tilde{x}^1\partial \tilde{x}^2} \right], \\
\tilde{g}_{12,2}&=\frac{-X(1+X^2)(1+Y^2)(Y^2[3X^2-Y^2]+1+X^2)(a+\Delta \epsilon z)^2}{\delta^6} \nonumber \\ &- \frac{2\Delta \epsilon XY(1+X^2)(1+Y^2)(a+\Delta \epsilon z)}{\delta^4} \frac{\partial z}{\partial \tilde{x}^2} + \epsilon^2 \left[ \frac{\partial z}{\partial \tilde{x}^2} \frac{\partial^2 z}{\partial \tilde{x}^1\partial \tilde{x}^2} + \frac{\partial z}{\partial \tilde{x}^1} \frac{\partial^2 z}{\partial \tilde{x}^2\partial \tilde{x}^2} \right], \\
\tilde{g}_{12,3}&=\frac{-2\Delta \epsilon XY(1+X^2)(1+Y^2)(a+\Delta \epsilon z)}{\delta^4} \frac{\partial z}{\partial \eta} + \epsilon^2 \left[ \frac{\partial z}{\partial \tilde{x}^2} \frac{\partial^2 z}{\partial \tilde{x}^1\partial \eta} + \frac{\partial z}{\partial \tilde{x}^1} \frac{\partial^2 z}{\partial \tilde{x}^2\partial \eta} \right], \\
\tilde{g}_{13,1}&=\epsilon^2 \left[ \frac{\partial^2 z}{\partial \tilde{x}^1 \partial \tilde{x}^1} \frac{\partial z}{\partial \eta} + \frac{\partial z}{\partial \tilde{x}^1} \frac{\partial^2 z}{\partial \tilde{x}^1 \partial \eta} \right], \\
\tilde{g}_{13,2}&=\epsilon^2 \left[ \frac{\partial^2 z}{\partial \tilde{x}^1 \partial \tilde{x}^2} \frac{\partial z}{\partial \eta} + \frac{\partial z}{\partial \tilde{x}^1} \frac{\partial^2 z}{\partial \tilde{x}^2 \partial \eta} \right], \\
\tilde{g}_{13,3}&=\epsilon^2 \left[ \frac{\partial^2 z}{\partial \tilde{x}^1 \partial \eta} \frac{\partial z}{\partial \eta} + \frac{\partial z}{\partial \tilde{x}^1} \frac{\partial^2 z}{\partial \eta^2} \right], \\
\tilde{g}_{22,1}&=\frac{2X(1+X^2)(1+Y^2)^2(Y^2-X^2-1)(a+\Delta \epsilon z)^2}{\delta^6} + \frac{2\Delta \epsilon(1+X^2)(1+Y^2)^2(a+\Delta \epsilon z)}{\delta^4} \frac{\partial z}{\partial \tilde{x}^1} \nonumber \\ &+ 2\epsilon^2 \frac{\partial z}{\partial \tilde{x}^2} \frac{\partial^2 z}{\partial \tilde{x}^1 \partial \tilde{x}^2}, \\
\tilde{g}_{22,2}&=\frac{4X^2Y(1+X^2)(1+Y^2)^2(a+\Delta \epsilon z)^2}{\delta^6} + 2\epsilon \frac{\partial z}{\partial \tilde{x}^2} \left[ \frac{\Delta (1+X^2)(1+Y^2)^2(a+\Delta \epsilon z)}{\delta^4} + \epsilon \frac{\partial^2 z}{\partial \tilde{x}^2 \partial \tilde{x}^2} \right], \\
\tilde{g}_{22,3}&=\frac{2\Delta \epsilon(1+X^2)(1+Y^2)^2(a+\Delta \epsilon z)}{\delta^4} \frac{\partial z}{\partial \eta} + 2\epsilon^2 \frac{\partial z}{\partial \tilde{x}^2} \frac{\partial^2 z}{\partial \tilde{x}^2 \partial \eta}, \\
\tilde{g}_{23,1}&=\epsilon^2 \left[ \frac{\partial^2 z}{\partial \tilde{x}^1 \partial \tilde{x}^2} \frac{\partial z}{\partial \eta} + \frac{\partial z}{\partial \tilde{x}^2} \frac{\partial^2 z}{\partial \tilde{x}^1 \partial \eta} \right], \\
\tilde{g}_{23,2}&=\epsilon^2 \left[ \frac{\partial^2 z}{\partial \tilde{x}^2 \partial \tilde{x}^2} \frac{\partial z}{\partial \eta} + \frac{\partial z}{\partial \tilde{x}^2} \frac{\partial^2 z}{\partial \tilde{x}^2 \partial \eta} \right], \\
\tilde{g}_{23,3}&=\epsilon^2 \left[ \frac{\partial^2 z}{\partial \tilde{x}^2 \partial \eta} \frac{\partial z}{\partial \eta} + \frac{\partial z}{\partial \tilde{x}^2} \frac{\partial^2 z}{\partial \eta^2} \right], \\
\tilde{g}_{33,1}&=2\epsilon^2 \frac{\partial z}{\partial \eta} \frac{\partial^2 z}{\partial \tilde{x}^1 \partial \eta}, \\
\tilde{g}_{33,2}&=2\epsilon^2 \frac{\partial z}{\partial \eta} \frac{\partial^2 z}{\partial \tilde{x}^2 \partial \eta}, \\
\tilde{g}_{33,3}&=2\epsilon^2 \frac{\partial z}{\partial \eta} \frac{\partial^2 z}{\partial \eta^2}.
\end{align}
For $\Delta=0$ and $\epsilon=1$:
\begin{align}
\tilde{\Gamma}_{111}&=\frac{2a^2XY^2(1+X^2)^2(1+Y^2)}{\delta^6} + \frac{\partial z}{\partial \tilde{x}^1} \frac{\partial^2 z}{\partial \tilde{x}^1 \partial \tilde{x}^1}, \\
\tilde{\Gamma}_{112}&=\frac{a^2Y(1+X^2)^2(1+Y^2)(X^2-Y^2-1)}{\delta^6}+ \frac{\partial z}{\partial \tilde{x}^1}\frac{\partial^2 z}{\partial \tilde{x}^1 \partial \tilde{x}^2}, \\
\tilde{\Gamma}_{113}&=\frac{\partial z}{\partial \tilde{x}^1}\frac{\partial^2 z}{\partial \tilde{x}^1 \partial \eta}, \\
\tilde{\Gamma}_{122}&=\frac{-2a^2X^3Y^2(1+X^2)(1+Y^2)}{\delta^6} + \frac{\partial z}{\partial \tilde{x}^1} \frac{\partial^2 z}{\partial \tilde{x}^2\partial \tilde{x}^2}, \\
\tilde{\Gamma}_{123}&=\frac{\partial z}{\partial \tilde{x}^1} \frac{\partial^2 z}{\partial \tilde{x}^2 \partial \eta}, \\
\tilde{\Gamma}_{133}&=\frac{\partial z}{\partial \tilde{x}^1} \frac{\partial^2 z}{\partial \eta^2}, \\
\tilde{\Gamma}_{211}&=\frac{-2a^2X^2Y^3(1+X^2)(1+Y^2)}{\delta^6} + \frac{\partial z}{\partial \tilde{x}^2} \frac{\partial^2 z}{\partial \tilde{x}^1\partial \tilde{x}^1}, \\
\tilde{\Gamma}_{212}&=\frac{a^2X(1+X^2)(1+Y^2)^2(Y^2-X^2-1)}{\delta^6} + \frac{\partial z}{\partial \tilde{x}^2} \frac{\partial^2 z}{\partial \tilde{x}^1 \partial \tilde{x}^2}, \\
\tilde{\Gamma}_{213}&=\frac{\partial z}{\partial \tilde{x}^2} \frac{\partial^2 z}{\partial \tilde{x}^1 \partial \eta}, \\
\tilde{\Gamma}_{222}&=\frac{2a^2X^2Y(1+X^2)(1+Y^2)^2}{\delta^6} + \frac{\partial z}{\partial \tilde{x}^2} \frac{\partial^2 z}{\partial \tilde{x}^2 \partial \tilde{x}^2}, \\
\tilde{\Gamma}_{223}&=\frac{\partial z}{\partial \tilde{x}^2} \frac{\partial^2 z}{\partial \tilde{x}^2 \partial \eta}, \\
\tilde{\Gamma}_{233}&=\frac{\partial z}{\partial \tilde{x}^2} \frac{\partial^2 z}{\partial \eta^2}, \\
\tilde{\Gamma}_{311}&=\frac{\partial^2 z}{\partial \tilde{x}^1 \partial \tilde{x}^1} \frac{\partial z}{\partial \eta}, \\
\tilde{\Gamma}_{312}&=\frac{\partial^2 z}{\partial \tilde{x}^1 \partial \tilde{x}^2} \frac{\partial z}{\partial \eta}, \\
\tilde{\Gamma}_{313}&=\frac{\partial^2 z}{\partial \tilde{x}^1 \partial \eta}\frac{\partial z}{\partial \eta}, \\
\tilde{\Gamma}_{322}&=\frac{\partial^2 z}{\partial \tilde{x}^2 \partial \tilde{x}^2} \frac{\partial z}{\partial \eta}, \\
\tilde{\Gamma}_{323}&=\frac{\partial^2 z}{\partial \tilde{x}^2 \partial \eta}\frac{\partial z}{\partial \eta}, \\
\tilde{\Gamma}_{333}&=\frac{\partial^2 z}{\partial \eta^2}\frac{\partial z}{\partial \eta}, \\
\tilde{\Gamma}^1_{11}&=\frac{2 X Y^2 }{ \delta^2 }, \\
\tilde{\Gamma}^1_{12}&=-\frac{ Y(1 + Y^2) }{ \delta^2 }=\Gamma^1_{21}, \\
\tilde{\Gamma}^1_{13}&=0=\tilde{\Gamma}^1_{31}, \\
\tilde{\Gamma}^1_{22}&=0, \\
\tilde{\Gamma}^1_{23}&=0=\tilde{\Gamma}^1_{32}, \\
\tilde{\Gamma}^1_{33}&=0,
\end{align}
\begin{align}
\tilde{\Gamma}^2_{11}&=0, \\
\tilde{\Gamma}^2_{12}&=-\frac{X(1+X^2)}{\delta^2}=\Gamma^2_{21}, \\
\tilde{\Gamma}^2_{13}&=0=\tilde{\Gamma}^2_{31}, \\
\tilde{\Gamma}^2_{22}&=\frac{2X^2Y}{\delta^2}, \\
\tilde{\Gamma}^2_{23}&=0=\tilde{\Gamma}^2_{32}, \\
\tilde{\Gamma}^2_{33}&=0,
\end{align}
\begin{align}
\tilde{\Gamma}^3_{11}&=\left(\frac{\partial z}{\partial \eta}\right)^{-1}\left[ \frac{\partial^2 z}{\partial \tilde{x}^1 \partial \tilde{x}^1} - \frac{2 X Y^2 }{ \delta^2 } \frac{\partial z}{\partial \tilde{x}^1}\right], \\
\tilde{\Gamma}^3_{12}&=\left(\frac{\partial z}{\partial \eta}\right)^{-1}\left[ \frac{\partial^2 z}{\partial \tilde{x}^1 \partial \tilde{x}^2} + \frac{ Y(1 + Y^2) }{ \delta^2 } \frac{\partial z}{\partial \tilde{x}^1} + \frac{X(1+X^2)}{\delta^2} \frac{\partial z}{\partial \tilde{x}^2}   \right]=\tilde{\Gamma}^3_{21}, \\
\tilde{\Gamma}^3_{13}&=\left(\frac{\partial z}{\partial \eta}\right)^{-1} \frac{\partial^2 z}{\partial \tilde{x}^1 \partial \eta}=\tilde{\Gamma}^3_{31}, \\
\tilde{\Gamma}^3_{22}&=\left(\frac{\partial z}{\partial \eta}\right)^{-1}\left[ \frac{\partial^2 z}{\partial \tilde{x}^2 \partial \tilde{x}^2} - \frac{2 X^2 Y }{ \delta^2 } \frac{\partial z}{\partial \tilde{x}^2}\right], \\
\tilde{\Gamma}^3_{23}&=\left(\frac{\partial z}{\partial \eta}\right)^{-1} \frac{\partial^2 z}{\partial \tilde{x}^2 \partial \eta}=\tilde{\Gamma}^3_{32}, \\
\tilde{\Gamma}^3_{33}&=\left(\frac{\partial z}{\partial \eta}\right)^{-1} \frac{\partial^2 z}{\partial \eta^2}.
\end{align}

For the 6 panels, the gravitational terms are
\begin{align}
\left( \Phi - \frac{1}{2} \tilde{g}_{00}\right)_{,1}&=\epsilon G \frac{\partial z}{\partial \tilde{x}^1}, \\
\left( \Phi - \frac{1}{2} \tilde{g}_{00}\right)_{,2}&=\epsilon G \frac{\partial z}{\partial \tilde{x}^2}, \\
\left( \Phi - \frac{1}{2} \tilde{g}_{00}\right)_{,3}&=\epsilon G \frac{\partial z}{\partial \eta}.
\end{align}
For panels $1$ to $4$, one calculates
\begin{align}
\tilde{g}_{01,1}&=\frac{2\Omega XY^2(1+X^2)(a+\Delta \epsilon z)^2}{\delta^4} + \frac{2\Omega \Delta \epsilon (1+X^2)(a+\Delta \epsilon z)}{\delta^2} \frac{\partial z}{\partial \tilde{x}^1}, \\
\tilde{g}_{01,2}&=\frac{-2\Omega Y(1+X^2)(1+Y^2)(a+\Delta \epsilon z)^2}{\delta^4} + \frac{2\Omega \Delta \epsilon (1+X^2)(a+\Delta \epsilon z)}{\delta^2} \frac{\partial z}{\partial \tilde{x}^2}, \\
\tilde{g}_{01,3}&=\frac{2\Omega \Delta \epsilon (1+X^2)(a+\Delta \epsilon z)}{\delta^2} \frac{\partial z}{\partial \eta}.
\end{align}
For $\Delta=0$ and $\epsilon=1$, one finds
\begin{align}
\tilde{\Gamma}_{101}&=0=\tilde{\Gamma}_{110}, \\
\tilde{\Gamma}_{102}&=\frac{-a^2\Omega Y(1+X^2)(1+Y^2)}{\delta^4}=\tilde{\Gamma}_{120}, \\
\tilde{\Gamma}_{103}&=0=\tilde{\Gamma}_{130}, \\
\tilde{\Gamma}_{201}&=\frac{a^2\Omega Y(1+X^2)(1+Y^2)}{\delta^4}=\tilde{\Gamma}_{210}, \\
\tilde{\Gamma}_{202}&=0=\tilde{\Gamma}_{220}, \\
\tilde{\Gamma}_{203}&=0=\tilde{\Gamma}_{230}, \\
\tilde{\Gamma}_{301}&=0=\tilde{\Gamma}_{310}, \\
\tilde{\Gamma}_{302}&=0=\tilde{\Gamma}_{320}, \\
\tilde{\Gamma}_{303}&=0=\tilde{\Gamma}_{330}, \\
\tilde{\Gamma}^1_{01}&=\frac{\Omega XY^2}{\delta^2}=\tilde{\Gamma}^1_{10}, \\
\tilde{\Gamma}^1_{02}&=-\frac{\Omega Y(1+Y^2)}{\delta^2}=\tilde{\Gamma}^1_{20}, \\
\tilde{\Gamma}^1_{03}&=0=\tilde{\Gamma}^1_{30}, \\
\tilde{\Gamma}^2_{01}&=\frac{\Omega Y(1+X^2)}{\delta^2}=\tilde{\Gamma}^2_{10}, \\
\tilde{\Gamma}^2_{02}&=-\frac{\Omega XY^2}{\delta^2}=\tilde{\Gamma}^2_{20}, \\
\tilde{\Gamma}^2_{03}&=0=\tilde{\Gamma}^2_{30}, \\
\tilde{\Gamma}^3_{01}&=-\left(\frac{\partial z}{\partial \eta}\right)^{-1}\left[ \frac{\Omega XY^2}{\delta^2} \frac{\partial z}{\partial \tilde{x}^1}+ \frac{\Omega Y(1+X^2)}{\delta^2} \frac{\partial z}{\partial \tilde{x}^2}  \right]=\tilde{\Gamma}^3_{10}, \\
\tilde{\Gamma}^3_{02}&=\left(\frac{\partial z}{\partial \eta}\right)^{-1}\left[ \frac{\Omega Y(1+Y^2)}{\delta^2} \frac{\partial z}{\partial \tilde{x}^1}+\frac{\Omega XY^2}{\delta^2}\frac{\partial z}{\partial \tilde{x}^2}  \right]=\tilde{\Gamma}^3_{20}, \\
\tilde{\Gamma}^3_{03}&=0=\tilde{\Gamma}^3_{30}.
\end{align}
For panel $5$, one calculates
\begin{align}
\tilde{g}_{01,1}&=\frac{-2\Omega XY^3(1+X^2)(a+\Delta \epsilon z)^2}{\delta^4} - \frac{2\Omega \Delta \epsilon Y(1+X^2)(a+\Delta \epsilon z)}{\delta^2} \frac{\partial z}{\partial \tilde{x}^1}, \\
\tilde{g}_{01,2}&=\frac{-\Omega (1+X^2)(1+Y^2)(1+X^2-Y^2)(a+\Delta \epsilon z)^2}{\delta^4} - \frac{2\Omega \Delta \epsilon Y(1+X^2)(a+\Delta \epsilon z)}{\delta^2} \frac{\partial z}{\partial \tilde{x}^2}, \\
\tilde{g}_{01,3}&=\frac{-2\Omega \Delta \epsilon Y(1+X^2)(a+\Delta \epsilon z)}{\delta^2} \frac{\partial z}{\partial \eta}, \\
\tilde{g}_{02,1}&=\frac{\Omega (1+X^2)(1+Y^2)(1-X^2+Y^2)(a+\Delta \epsilon z)^2}{\delta^4} + \frac{2\Omega \Delta \epsilon X(1+Y^2)(a+\Delta \epsilon z)}{\delta^2} \frac{\partial z}{\partial \tilde{x}^1}, \\
\tilde{g}_{02,2}&=\frac{2\Omega X^3Y(1+Y^2)(a+\Delta \epsilon z)^2}{\delta^4} + \frac{2\Omega \Delta \epsilon X(1+Y^2)(a+\Delta \epsilon z)}{\delta^2} \frac{\partial z}{\partial \tilde{x}^2}, \\
\tilde{g}_{02,3}&=\frac{2\Omega \Delta \epsilon X(1+Y^2)(a+\Delta \epsilon z)}{\delta^2} \frac{\partial z}{\partial \eta}.
\end{align}
For $\Delta=0$ and $\epsilon=1$, one finds
\begin{align}
\tilde{\Gamma}_{101}&=0=\tilde{\Gamma}_{110}, \\
\tilde{\Gamma}_{102}&=\frac{-a^2\Omega (1+X^2)(1+Y^2)}{\delta^4}=\tilde{\Gamma}_{120}, \\
\tilde{\Gamma}_{103}&=0=\tilde{\Gamma}_{130}, \\
\tilde{\Gamma}_{201}&=\frac{a^2\Omega (1+X^2)(1+Y^2)}{\delta^4}=\tilde{\Gamma}_{210}, \\
\tilde{\Gamma}_{202}&=0=\tilde{\Gamma}_{220}, \\
\tilde{\Gamma}_{203}&=0=\tilde{\Gamma}_{230}, \\
\tilde{\Gamma}_{301}&=0=\tilde{\Gamma}_{310}, \\
\tilde{\Gamma}_{302}&=0=\tilde{\Gamma}_{320}, \\
\tilde{\Gamma}_{303}&=0=\tilde{\Gamma}_{330}, \\
\tilde{\Gamma}^1_{01}&=\frac{\Omega XY}{\delta^2}=\tilde{\Gamma}^1_{10}, \\
\tilde{\Gamma}^1_{02}&=-\frac{\Omega (1+Y^2)}{\delta^2}=\tilde{\Gamma}^1_{20}, \\
\tilde{\Gamma}^1_{03}&=0=\tilde{\Gamma}^1_{30}, \\
\tilde{\Gamma}^2_{01}&=\frac{\Omega (1+X^2)}{\delta^2}=\tilde{\Gamma}^2_{10}, \\
\tilde{\Gamma}^2_{02}&=-\frac{\Omega XY}{\delta^2}=\tilde{\Gamma}^2_{20}, \\
\tilde{\Gamma}^2_{03}&=0=\tilde{\Gamma}^2_{30}, \\
\tilde{\Gamma}^3_{01}&=-\left(\frac{\partial z}{\partial \eta}\right)^{-1}\left[ \frac{\Omega XY}{\delta^2} \frac{\partial z}{\partial \tilde{x}^1}+ \frac{\Omega (1+X^2)}{\delta^2} \frac{\partial z}{\partial \tilde{x}^2}  \right]=\tilde{\Gamma}^3_{10}, \\
\tilde{\Gamma}^3_{02}&=\left(\frac{\partial z}{\partial \eta}\right)^{-1}\left[ \frac{\Omega (1+Y^2)}{\delta^2} \frac{\partial z}{\partial \tilde{x}^1}+\frac{\Omega XY}{\delta^2}\frac{\partial z}{\partial \tilde{x}^2}  \right]=\tilde{\Gamma}^3_{20}, \\
\tilde{\Gamma}^3_{03}&=0=\tilde{\Gamma}^3_{30}.
\end{align}
For panel $6$, one calculates
\begin{align}
\tilde{g}_{01,1}&=\frac{2\Omega XY^3(1+X^2)(a+\Delta \epsilon z)^2}{\delta^4} + \frac{2\Omega \Delta \epsilon Y(1+X^2)(a+\Delta \epsilon z)}{\delta^2} \frac{\partial z}{\partial \tilde{x}^1}, \\
\tilde{g}_{01,2}&=\frac{\Omega (1+X^2)(1+Y^2)(1+X^2-Y^2)(a+\Delta \epsilon z)^2}{\delta^4} + \frac{2\Omega \Delta \epsilon Y(1+X^2)(a+\Delta \epsilon z)}{\delta^2} \frac{\partial z}{\partial \tilde{x}^2}, \\
\tilde{g}_{01,3}&=\frac{2\Omega \Delta \epsilon Y(1+X^2)(a+\Delta \epsilon z)}{\delta^2} \frac{\partial z}{\partial \eta}, \\
\tilde{g}_{02,1}&=\frac{-\Omega (1+X^2)(1+Y^2)(1-X^2+Y^2)(a+\Delta \epsilon z)^2}{\delta^4} - \frac{2\Omega \Delta \epsilon X(1+Y^2)(a+\Delta \epsilon z)}{\delta^2} \frac{\partial z}{\partial \tilde{x}^1}, \\
\tilde{g}_{02,2}&=\frac{-2\Omega X^3Y(1+Y^2)(a+\Delta \epsilon z)^2}{\delta^4} - \frac{2\Omega \Delta \epsilon X(1+Y^2)(a+\Delta \epsilon z)}{\delta^2} \frac{\partial z}{\partial \tilde{x}^2}, \\
\tilde{g}_{02,3}&=\frac{-2\Omega \Delta \epsilon X(1+Y^2)(a+\Delta \epsilon z)}{\delta^2} \frac{\partial z}{\partial \eta}.
\end{align}
For $\Delta=0$ and $\epsilon=1$, one finds
\begin{align}
\tilde{\Gamma}_{101}&=0=\tilde{\Gamma}_{110}, \\
\tilde{\Gamma}_{102}&=\frac{a^2\Omega (1+X^2)(1+Y^2)}{\delta^4}=\tilde{\Gamma}_{120}, \\
\tilde{\Gamma}_{103}&=0=\tilde{\Gamma}_{130}, \\
\tilde{\Gamma}_{201}&=\frac{-a^2\Omega (1+X^2)(1+Y^2)}{\delta^4}=\tilde{\Gamma}_{210}, \\
\tilde{\Gamma}_{202}&=0=\tilde{\Gamma}_{220}, \\
\tilde{\Gamma}_{203}&=0=\tilde{\Gamma}_{230}, \\
\tilde{\Gamma}_{301}&=0=\tilde{\Gamma}_{310}, \\
\tilde{\Gamma}_{302}&=0=\tilde{\Gamma}_{320}, \\
\tilde{\Gamma}_{303}&=0=\tilde{\Gamma}_{330}, \\
\tilde{\Gamma}^1_{01}&=-\frac{\Omega XY}{\delta^2}=\tilde{\Gamma}^1_{10}, \\
\tilde{\Gamma}^1_{02}&=\frac{\Omega (1+Y^2)}{\delta^2}=\tilde{\Gamma}^1_{20}, \\
\tilde{\Gamma}^1_{03}&=0=\tilde{\Gamma}^1_{30}, \\
\tilde{\Gamma}^2_{01}&=-\frac{\Omega (1+X^2)}{\delta^2}=\tilde{\Gamma}^2_{10}, \\
\tilde{\Gamma}^2_{02}&=\frac{\Omega XY}{\delta^2}=\tilde{\Gamma}^2_{20}, \\
\tilde{\Gamma}^2_{03}&=0=\tilde{\Gamma}^2_{30}, \\
\tilde{\Gamma}^3_{01}&=\left(\frac{\partial z}{\partial \eta}\right)^{-1}\left[ \frac{\Omega XY}{\delta^2} \frac{\partial z}{\partial \tilde{x}^1}+ \frac{\Omega (1+X^2)}{\delta^2} \frac{\partial z}{\partial \tilde{x}^2}  \right]=\tilde{\Gamma}^3_{10}, \\
\tilde{\Gamma}^3_{02}&=-\left(\frac{\partial z}{\partial \eta}\right)^{-1}\left[ \frac{\Omega (1+Y^2)}{\delta^2} \frac{\partial z}{\partial \tilde{x}^1}+\frac{\Omega XY}{\delta^2}\frac{\partial z}{\partial \tilde{x}^2}  \right]=\tilde{\Gamma}^3_{20}, \\
\tilde{\Gamma}^3_{03}&=0=\tilde{\Gamma}^3_{30}.
\end{align}

One or two vertical velocities may be used. If one vertical velocity is chosen ($\tilde{u}^3=\dot\eta$), then the $\tilde{\Gamma}^3_{\mu\nu}$ become useful. Otherwise, if two vertical velocities are chosen, then $\tilde{u}^3$ is diagnosed from the other vertical velocity ($w=\dot z$):
\begin{align}
\tilde{u}^3=\left(\frac{\partial z}{\partial \eta}\right)^{-1}\left[ w - \frac{\partial z}{\partial \tilde{x}^1} \tilde{u}^1 - \frac{\partial z}{\partial \tilde{x}^2} \tilde{u}^2 \right]
\end{align}
and the vertical momentum equation is written (spherical geometry, shallow and non-hydrostatic atmosphere):
\begin{align}
\frac{\partial (\sqrt{\tilde{g}}\rho \tilde{u}^\mu w)}{\partial \tilde{x}^\mu}=-\sqrt{\tilde{g}} \left(\frac{\partial z}{\partial \eta}\right)^{-1} \frac{\partial p}{\partial\eta}-\sqrt{\tilde{g}}\rho g_r,
\end{align}
where $g_r$ is the constant $9.81$ m s$^{-2}$ and $\sqrt{\tilde{g}}$ is provided by \eqref{rgt} with $\Delta=0$ and $\epsilon=1$.

Under the spherical geopotential and shallow-atmosphere approximations, the inviscid horizontal momentum equations are ($i=1,2$):
\begin{align}
\left( \sqrt{\tilde{g}}\rho \tilde{u}^i\tilde{u}^\nu\right)_{,\nu} = - \sqrt{\tilde{g}}\tilde{h}^{ij}p_{,j} -2\sqrt{\tilde{g}} \rho \tilde{\Gamma}^i_{j0} \tilde{u}^j - \sqrt{\tilde{g}} \rho \tilde{\Gamma}^i_{jk}\tilde{u}^j\tilde{u}^k,
\end{align}
while the inviscid vertical momentum equation with one vertical velocity is
\begin{align}
\left( \sqrt{\tilde{g}}\rho \tilde{u}^3\tilde{u}^\nu\right)_{,\nu} = - \sqrt{\tilde{g}} \rho \left(\frac{\partial z}{\partial \eta}\right)^{-1} g_r - \sqrt{\tilde{g}}\tilde{h}^{3j}p_{,j} -2\sqrt{\tilde{g}} \rho \tilde{\Gamma}^3_{j0} \tilde{u}^j - \sqrt{\tilde{g}} \rho \tilde{\Gamma}^3_{jk}\tilde{u}^j\tilde{u}^k.
\end{align}

At the interface of two panels, consistency conditions on 4-vectors may be established. At the interface of panels $(p,q)=(1,2), (2,3),(3,4),(4,1)$, one obtains
\begin{align}
A^0_{(p)}&=A^0_{(q)}, \\
A^1_{(p)}&=A^1_{(q)}, \\
A^2_{(p)}&=\frac{2Y}{1+Y^2} A^1_{(q)} + A^2_{(q)}, \\
A^3_{(p)}&=A^3_{(q)}, \\
A_{(p)0}&=A_{(q)0}, \\
A_{(p)1}&=A_{(q)1}-\frac{2Y}{1+Y^2} A_{(q)2}, \\
A_{(p)2}&=A_{(q)2}, \\
A_{(p)3}&=A_{(q)3}.
\end{align}
At the interface of panels $(5,1)$, one obtains
\begin{align}
A^0_{(5)}&=A^0_{(1)}, \\
A^1_{(5)}&=A^1_{(1)}-\frac{2X}{1+X^2} A^2_{(1)}, \\
A^2_{(5)}&=A^2_{(1)}, \\
A^3_{(5)}&=A^3_{(1)}, \\
A_{(5)0}&=A_{(1)0}, \\
A_{(5)1}&=A_{(1)1}, \\
A_{(5)2}&=\frac{2X}{1+X^2}A_{(1)1} + A_{(1)2}, \\
A_{(5)3}&=A_{(1)3},
\end{align}
with $X$ defined on panel $1$. At the interface of panels $(5,2)$, one obtains
\begin{align}
A^0_{(5)}&=A^0_{(2)}, \\
A^1_{(5)}&=-A^2_{(2)}, \\
A^2_{(5)}&=A^1_{(2)}-\frac{2X}{1+X^2} A^2_{(2)}, \\
A^3_{(5)}&=A^3_{(2)}, \\
A_{(5)0}&=A_{(2)0}, \\
A_{(5)1}&=-\frac{2X}{1+X^2} A_{(2)1} - A_{(2)2}, \\
A_{(5)2}&=A_{(2)1}, \\
A_{(5)3}&=A_{(2)3},
\end{align}
with $X$ defined on panel $2$. At the interface of panels $(5,3)$, one obtains
\begin{align}
A^0_{(5)}&=A^0_{(3)}, \\
A^1_{(5)}&=- A^1_{(3)} + \frac{2X}{1+X^2} A^2_{(3)}, \\
A^2_{(5)}&=-A^2_{(3)}, \\
A^3_{(5)}&=A^3_{(3)}, \\
A_{(5)0}&=A_{(3)0}, \\
A_{(5)1}&=-A_{(3)1}, \\
A_{(5)2}&=-\frac{2X}{1+X^2} A_{(3)1} - A_{(3)2}, \\
A_{(5)3}&=A_{(3)3},
\end{align}
with $X$ defined on panel $3$. At the interface of panels $(5,4)$, one obtains
\begin{align}
A^0_{(5)}&=A^0_{(4)}, \\
A^1_{(5)}&=A^2_{(4)}, \\
A^2_{(5)}&=- A^1_{(4)} + \frac{2X}{1+X^2} A^2_{(4)}, \\
A^3_{(5)}&=A^3_{(4)}, \\
A_{(5)0}&=A_{(4)0}, \\
A_{(5)1}&=\frac{2X}{1+X^2} A_{(4)1} + A_{(4)2}, \\
A_{(5)2}&=-A_{(4)1}, \\
A_{(5)3}&=A_{(4)3},
\end{align}
with $X$ defined on panel $4$. At the interface of panels $(6,1)$, one obtains
\begin{align}
A^0_{(6)}&=A^0_{(1)}, \\
A^1_{(6)}&=A^1_{(1)} + \frac{2X}{1+X^2} A^2_{(1)}, \\
A^2_{(6)}&=A^2_{(1)}, \\
A^3_{(6)}&=A^3_{(1)}, \\
A_{(6)0}&=A_{(1)0}, \\
A_{(6)1}&=A_{(1)1}, \\
A_{(6)2}&=-\frac{2X}{1+X^2} A_{(1)1}+A_{(1)2}, \\
A_{(6)3}&=A_{(1)3},
\end{align}
with $X$ defined on panel $1$. At the interface of panels $(6,2)$, one obtains
\begin{align}
A^0_{(6)}&=A^0_{(2)}, \\
A^1_{(6)}&=A^2_{(2)}, \\
A^2_{(6)}&=- A^1_{(2)}-\frac{2X}{1+X^2} A^2_{(2)}, \\
A^3_{(6)}&=A^3_{(2)}, \\
A_{(6)0}&=A_{(2)0}, \\
A_{(6)1}&=-\frac{2X}{1+X^2} A_{(2)1} + A_{(2)2}, \\
A_{(6)2}&=-A_{(2)1}, \\
A_{(6)3}&=A_{(2)3},
\end{align}
with $X$ defined on panel $2$. At the interface of panels $(6,3)$, one obtains
\begin{align}
A^0_{(6)}&=A^0_{(3)}, \\
A^1_{(6)}&=- A^1_{(3)}-\frac{2X}{1+X^2} A^2_{(3)}, \\
A^2_{(6)}&=-A^2_{(3)}, \\
A^3_{(6)}&=A^3_{(3)}, \\
A_{(6)0}&=A_{(3)0}, \\
A_{(6)1}&=-A_{(3)1}, \\
A_{(6)2}&=\frac{2X}{1+X^2} A_{(3)1} - A_{(3)2}, \\
A_{(6)3}&=A_{(3)3},
\end{align}
with $X$ defined on panel $3$. At the interface of panels $(6,4)$, one obtains
\begin{align}
A^0_{(6)}&=A^0_{(4)}, \\
A^1_{(6)}&=-A^2_{(4)}, \\
A^2_{(6)}&=A^1_{(4)} + \frac{2X}{1+X^2} A^2_{(4)}, \\
A^3_{(6)}&=A^3_{(4)}, \\
A_{(6)0}&=A_{(4)0}, \\
A_{(6)1}&=\frac{2X}{1+X^2} A_{(4)1} - A_{(4)2}, \\
A_{(6)2}&=A_{(4)1}, \\
A_{(6)3}&=A_{(4)3},
\end{align}
with $X$ defined on panel $4$.

%The Christoffel symbols are provided for the shallow atmosphere in 2D. For panels $p=1,2,3,4$, one has
%\begin{align}
%   \Gamma^1_{01}&= \frac{ \Omega   X   Y^{2} }{ \delta^{2} }=\Gamma^1_{10}, \\
%   \Gamma^1_{02}&=- \frac{ \Omega  Y(1 + Y^{2}) }{ \delta^{2} }=\Gamma^1_{20}, \\
%   \Gamma^2_{01}&= \frac{ \Omega   Y   (1 + X^{2}) }{ \delta^2 }=\Gamma^2_{10}, \\
%   \Gamma^2_{02}&=- \frac{ \Omega   X   Y^{2} }{ \delta^2 }=\Gamma^2_{20}.
%\end{align}
%For panel 5, one has
%\begin{align}
%   \Gamma^1_{01}&= \frac{ \Omega X Y }{ \delta^2 }=\Gamma^1_{10}, \\
%   \Gamma^1_{02}&=-\frac{ \Omega  (1 + Y^2) }{ \delta^2 }=\Gamma^1_{20}, \\
%   \Gamma^2_{01}&= \frac{ \Omega (1 + X^2) }{ \delta^2 }=\Gamma^2_{10}, \\
%   \Gamma^2_{02}&=-\frac{ \Omega X Y }{ \delta^2 }=\Gamma^2_{20}.
%\end{align}
%For panel 6, one has
%\begin{align}
%   \Gamma^1_{01}&=-\frac{ \Omega X Y }{ \delta^2 }=\Gamma^1_{10}, \\
%   \Gamma^1_{02}&= \frac{ \Omega  (1 + Y^2) }{ \delta^2 }=\Gamma^1_{20}, \\
%   \Gamma^2_{01}&=-\frac{ \Omega (1 + X^2) }{ \delta^2 }=\Gamma^2_{10}, \\
%   \Gamma^2_{02}&= \frac{ \Omega X Y }{ \delta^2 }=\Gamma^2_{20}.
%\end{align}
%The remaining Christoffel symbols are the same on all panels
%\begin{align}
%   \Gamma^1_{11}&= \frac{2 X Y^2 }{ \delta^2 }, \\
%   \Gamma^1_{12}&= -\frac{ Y(1 + Y^2) }{ \delta^2 }=\Gamma^1_{21}, \\
%   \Gamma^1_{22}&= 0, \\
%   \Gamma^2_{11}&= 0, \\
%   \Gamma^2_{12}&= - \frac{ X (1 + X^2) }{ \delta^2 }=\Gamma^2_{21}, \\
%   \Gamma^2_{22}&= \frac{ 2 X^2 Y }{ \delta^2 }.
%\end{align}

\end{document}
