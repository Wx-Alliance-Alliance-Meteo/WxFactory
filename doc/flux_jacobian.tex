\documentclass[12pt]{article}
\usepackage{lineno}
\usepackage{array}
\usepackage{amsmath}
\usepackage{amssymb}
\usepackage{amsthm}
\usepackage{algorithm}
\usepackage{algorithmic}
\usepackage{booktabs}
%\usepackage{subfigure}
\usepackage{caption}
\usepackage{subcaption}
\usepackage{enumitem}
\usepackage{color}
\usepackage[title]{appendix}

\title{Flux Jacobian of the Euler equations in tensor form}
\author{Stéphane Gaudreault}
\date{\today}

\begin{document}
\maketitle

\begin{abstract}
Flux jacobian are derived.
\end{abstract}

\section{Results}

The components $(\nu,\alpha)$ of the three flux Jacobian matrices ${\cal J}^\mu$ are defined as
\begin{align}
\left( {\cal J}^\mu \right)^\nu_{\;\;\alpha}\equiv \frac{\partial (\sqrt{g}\,T^{\mu\nu})}{\partial q^\alpha}.
\end{align}
Note that the left-hand side of the Euler equations may be rewritten with the flux Jacobian matrices as
\begin{align}
\frac{\partial}{\partial x^\nu}(\sqrt{g}\,T^{\mu\nu})=\left( {\cal J}^\mu \right)^\nu_{\;\;\alpha} \frac{\partial q^\alpha}{\partial x^\nu}.
\end{align}
They are written explicitly as ${\cal J}^0=I_{5\times 5}$ (the identity matrix) and
\begin{align}
{\cal J}^1=\left(\begin{array}{ccccc} 0 & 1 & 0 & 0 & 0\\ h^{11} c^2 -u^{1}\,u^{1} & 2\,u^{1} & 0 & 0 & 0\\ h^{21} c^2 -u^{1}\,u^{2} & u^{2} & u^{1} & 0 & 0\\ h^{31} c^2 -u^{1}\,u^{3} & u^{3} & 0 & u^{1} & 0\\ -\theta \,u^{1} & \theta  & 0 & 0 & u^{1} \end{array}\right) \\
{\cal J}^2=\left(\begin{array}{ccccc} 0 & 0 & 1 & 0 & 0\\ h^{12} c^2-u^{1}\,u^{2} & u^{2} & u^{1} & 0 & 0\\ h^{22} c^2 - u^{2}\,u^{2} & 0 & 2\,u^{2} & 0 & 0\\ h^{32} c^2 - u^{3}\,u^{2} & 0 & u^{3} & u^{2} & 0\\ -\theta \,u^{2} & 0 & \theta  & 0 & u^{2} \end{array}\right)\\
{\cal J}^3=\left(\begin{array}{ccccc} 0 & 0 & 0 & 1 & 0\\ h^{13} c^2 -u^{1}\,u^{3} & u^{3} & 0 & u^{1} & 0\\ h^{23} c^2 -u^{2}\,u^{3} & 0 & u^{3} & u^{2} & 0\\ h^{33} c^2 - u^{3}\,u^{3} & 0 & 0 & 2\,u^{3} & 0\\ -\theta \,u^{3} & 0 & 0 & \theta  & u^{3} \end{array}\right)
\end{align}
where $c = \sqrt{\left(\frac{\partial P}{\partial\rho}\right)_s} = \sqrt{\frac{\gamma P}{\rho}}$ is the speed of sound, $\gamma$ is the heat capacity ratio for dry air and $s$ is the entropy.
The five eigenvalues of ${\cal J}^0$ correspond to $u^0=1$. The five eigenvalues of ${\cal J}^1$ are
\begin{equation}\label{eq:eig_flux1}
u^1, \, u^1, \,u^1, \,u^1 \pm \sqrt{h^{11}} c,
\end{equation}
those of ${\cal J}^2$ are
\begin{equation}\label{eq:eig_flux2}
u^2, \, u^2, \, u^2, \, u^2 \pm \sqrt{h^{22}} c
\end{equation}
and, for ${\cal J}^3$, one obtain
\begin{equation}\label{eq:eig_flux3}
u^3, \, u^3, \, u^3, \, u^3 \pm \sqrt{h^{33}} c.
\end{equation}
The intrinsic accoustic wave velocities are deduced from these eigenvalues as $a^1 \equiv \sqrt{h^{11}} c$, $a^2 \equiv \sqrt{h^{22}} c$ and $a^3 \equiv \sqrt{h^{33}} c$.


\end{document}
